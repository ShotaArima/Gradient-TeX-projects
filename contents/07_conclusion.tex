\subsection{まとめ}
本研究では,入力データの特徴とその重要度を考慮した敵対的サンプル生成手法について提案を行った.従来手法は,表形式データに対するノイズの付加を行う画期的な手法であったが,特定の特徴量に対して過剰なノイズが付与される問題が存在し,離散値を持つ表形式データに対して連続値の敵対的サンプルが生成される問題があった.これらの問題に対処するため,本研究では特徴量の重要度算出方法を改善し,出力データの離散化を行うことで,より自然な敵対的サンプルの生成を実現した.

具体的には,従来手法で用いられていた相関係数の正規化処理を見直し,重要度算出方法を改善することで,特定の特徴量に対する過剰なノイズの付与を抑制した.また,出力データの離散化手法として四捨五入およびランダム離散化を導入し,元データの特性を保持しつつ,現実的な敵対的サンプルを生成することができた.

実験結果から,提案手法は従来手法に比べて,元データに対する変更が少なく,より自然な敵対的サンプルを生成できることが示された.特に,重要度算出法( $\bm{v}_{\mathrm{sqrt}}$ )と四捨五入による離散化手法の組み合わせが最も効果的であり,元データの特徴を保持しつつ,適切なノイズを付加することで,より自然な敵対的サンプルを生成することができた.

さらに,提案手法は従来手法に比べて,成功率および平均距離の指標で優れた性能を示した.これにより,提案手法が従来手法に比べて,より効果的に敵対的サンプルを生成できることが確認された.一方で,重み距離の指標においては,提案手法( $\bm{v}_{\mathrm{sqrt}}$ )が劣る結果となったが,これは重要度算出アルゴリズムにおける平方根操作の影響によるものであると考えられる.

本研究の成果は,機械学習モデルの堅牢性を向上させるための敵対的学習の重要な礎となるより精度の高い敵対的サンプルの生成を行うことができた,今後の研究においても,提案手法をさらに改良し,より広範なデータセットやモデルに適用することで,その有効性を検証していくことが期待される.また,提案手法を実運用環境に適用することで,機械学習モデルの安全性を向上させ,敵対的サンプルによる攻撃からシステムを保護することが可能となる.

\subsection{今後の課題}
本研究では,使用したデータセットのみに適用を行った.実験結果として他のデータセットに対しても提案手法を適用することで,汎用性の高い手法であるかを検証する必要がある.また,提案手法のハイパーパラメータの調整や,他の敵対的サンプル生成手法との比較を行うことで,提案手法の有効性を評価する必要がある.さらに,今回の実験の中で生成された敵対的サンプルについてデータによっては評価が良くない場合もあったが,異なるデータセットにおいて良い評価を得ることができたことから,複数の敵対的サンプル生成手法を組み合わせるアンサンブル学習などが効果的である可能性がある.この点についても今後の研究で調査を行うべきである.
