\subsection{まとめ}
本研究では,入力データの特徴とその重要度を考慮した敵対的サンプル生成手法について提案を行った.従来手法は,表形式データに対するノイズの付加を行う画期的な手法であったが,特定の特徴量に対して過剰なノイズが付与される問題が存在し,離散値を持つ表形式データに対して連続値の敵対的サンプルが生成される問題があった.そこで,特徴量の重要度算出方法を改善し,出力データの離散化を行うことで,より自然な敵対的サンプルの生成を実現した.実験結果から,提案手法により過度なノイズの付与を抑制でき,より自然な敵対的サンプルの生成が可能になった.
\subsection{今後の課題}
本研究では,使用したデータセットのみに適用を行った.実験結果として他のデータセットに対しても提案手法を適用することで,汎用性の高い手法であるかを検証する必要がある.また,提案手法のハイパーパラメータの調整や,他の敵対的サンプル生成手法との比較を行うことで,提案手法の有効性を評価する必要がある.さらに,今回の実験の中で生成された敵対的サンプルについてデータによっては評価が良くない場合もあったが,異なるデータセットにおいて良い評価を得ることができたことから,複数の敵対的サンプル生成手法を組み合わせるアンサンブル学習などが効果的である可能性がある.この点についても今後の研究で調査を行うべきである.
