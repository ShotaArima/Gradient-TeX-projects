\subsection{提案手法の概要}
本研究では,従来手法における2つの主要な課題を解決することを目的とする.第一に従来手法では,特定の特徴量に対する過剰なノイズが付与されていることがあり.敵対的サンプルがモデルの挙動を適切に評価する妨げとなる場合がある.第二に,元データの特徴量が離散値であるにも関わらず,連続値として敵対的サンプルが生成されている問題がある.このような課題を解決するため,本研究では特徴量に対する「重要度算出法の改善」と「出力データの離散化」という2つの提案を行う.まず,重要度算出法の改善により,特徴量ごとの重要度の偏りを軽減し,モデルの特性に応じた適切な敵対的サンプルを生成できるようにする.次に,出力データの離散化手法を導入し,元データの特性を保持しつつ,生成された敵対的サンプルを離散値として変換することで実用性を向上させる.

\subsection{重要度算出法の改善}
重要度算出法における課題として,相関係数の絶対値を正規化することで元データの分布の特徴をうまく捉えていないのではという可能性があった.そこで本研究では,この課題を解決するために以下の2つの新しい算出式を提案する.

\autoequation{\bm{v_{\mathrm{raw}}} = \|\rho_{\bm{X},Y}\|}
\autoequation{\bm{v_{\mathrm{sqrt}}} = \sqrt{\cfrac{\|\rho_{\bm{X},Y}\|}{\sum^{d}_{i=1}{(\|\rho_{\bm{X_i},Y}\|)^2}}}}

ここで, $\bm{X}$ は特徴量ベクトル, $\bm{Y}$ は目的変数, $\rho_{\bm{X},Y}$ は特徴量 $\bm{X}$ と目的変数 $\bm{Y}$ の相関係数を表す.また,$\bm{X_i}$ は $i$ 番目の特徴量,$d$ は特徴量の数を表す.

式(7)では,特徴量 $\bm{X}$ と目的変数 $\bm{Y}$ の相関係数の絶対値を使用して重要度を算出する.この式では従来手法と異なり正規化の処理を外すことで,特定の特徴量に過剰なノイズが加わるリスクを軽減することができる.

% 従来手法では,各特徴量について相関係数の絶対値を正規化した算出法(式(6))であった.この算出法はデータの特性に合わせた重要度を算出することができている.しかし,今回のデータセットの場合,一番重要度の高い特徴量と二番目以降に重要度の高い特徴量で差が大きく,一番を大きい特徴量に対して強いノイズの回避傾向がみて取れる.よって二番目に重要度の大きい特徴量に対して大きなノイズを加えてしまうような状態となっている.相関係数というのは-1~1までの値である変数と変数の関係を表現したものである.今回はそれを絶対値化しているため0~1までの値で表現ができる.この関係はすでにスケーリングされており,正規化する必要がないと考えたため,式(7)では正規化を行わずに重要度を算出している.
式(8)では,従来手法の重要度算出法に平方根を導入し,大きな重要度を相対的に小さくし,小さな重要度を相対的に大きくする特性を持たせている.これは,正規化によってより極端に重要度は大きくなってしまう問題を和らげるために導入する.平方根は,大きい値を小さく,小さい値を大きくし全体的に変化を抑えるような効果がある.
これらの算出式は,データセットの特性に適応しながら,従来の手法で発生していた課題を解消することを目的としている.

\subsection{出力データの離散化手法}
提案手法で生成される敵対的サンプルは,元データの離散値特性を反映する必要がある.これを実現するため,本研究では以下の2つの離散化手法を提案する.

一つ目は,小数部分を四捨五入することで連続値を離散値に変換する方法である.この方法は,一般的に用いられる手法であり,簡易かつ高速に変換が可能である.しかし,小数点以下の情報が損失するため,敵対的サンプルの多様性が制限される可能性がある.

二つ目は,ランダム性を取り入れた確率的な離散化手法である,この方法では,小数部分の数値を確率として解釈し,その確率に基づいて繰り上げまたは繰り下げを行う.例えば,0.4という値に対しては40\%で繰り上げを行い,60\%の確率で繰り下げを行う.このアプローチにより,少数部分の情報を活用しつつ,ノイズの影響を反映することができる.

これら2つの手法について,実験的に比較検証を行い,最適な離散化手法を決定することで,提案手法の有効性をさらに向上させる.