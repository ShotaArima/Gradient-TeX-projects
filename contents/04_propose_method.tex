\subsection{提案手法の概要}
前述の通り,従来手法では特定の特徴量に対する過剰なノイズが付与されている問題が存在する.また元データの特徴量が離散値であるにも限らず,連続値として敵対的サンプルが生成されている.この解決のため,特徴量に対する重要度算出法の改善と出力データの離散化を行う.
\subsection{重要度算出法の改善}
重要度算出法として,元データの重要度を維持しつつ,特定の特徴量の重要度が大きくなることを防ぐため,以下の式を提案する.

\autoequation{\bm{v}' = \|\rho_{\bm{X},Y}\|}
\autoequation{\bm{v}'' = \sqrt{\cfrac{\|\rho_{\bm{X},Y}\|}{\sum^{d}_{i=1}{(\|\rho_{\bm{X_i},Y}\|)^2}}}}

式(7)では,特徴量 $\bm{X}$ と目的変数 $\bm{Y}$ の相関係数の絶対値を用いる.これは従来手法で行っている重要度の算出から正規化の処理を外したものである.
式(8)では,従来手法の重要度算出法に平方根したものである.
従来手法では,各特徴量について相関係数の絶対値を正規化したものを使用していた.この算出法はデータの特性に合わせた重要度を算出することができている.しかし,今回のデータセットの場合前述の通り,一番重要度の高い特徴量と二番目以降に重要度の高い特徴量でさが大きく,一番を大きい特徴量に対して強いノイズの回避傾向がみて取れる.よって二番目に重要度の大きい特徴量に対して大きなノイズを加えてしまうような状態となっている.よってこれらの問題を解決するため,式(7)では正規化処理を外し,式(8)では平方根を導入した.相関係数というのは-1~1までの値である変数と変数の関係を表現したものである.よって正規化を行う必要はそこまでなく返って今回のような状態を引き起こしてしまう可能性がある.また,式(8)で平方根を導入した経緯として,小さい値ほど大きく,大きい値ほど小さくする特性がある.これにより,ノイズを強く回避する特徴を和らげることができ,より自然な敵対的サンプルを生成できる可能性がある.

\subsection{出力データの離散化手法}

先ほどの重要度特徴量算出により生成された敵対的サンプルは連続値であるため離散化の提案を行う.
一つ目に,小数部分の四捨五入による整数化である.この方法は,連続値を離散値に変換する際に最も一般的な方法であり,世間一般に広く使われている.しかし,四捨五入はその小数第一位,二位の値によって固定した変更が行われてしまい,情報損失の可能性がある.そこで二つ目にノイズが加わったノイズにランダム性を取り入れ,連続値を離散値に変換する方法を提案する.たとえば,ある特徴量が0である元データに対してノイズが加わり,0.4となった場合,四捨五入を行うと0に切り捨てられてしまいノイズが加わった効果を得ることができない.しかし,小数部分の数値を確率で見ることで40\%で繰り上げを行うことで,ノイズの損失を抑えることができる.よってこれらの2つの方法を実験で比較し,最適な離散化手法を提案する.