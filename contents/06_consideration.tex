% \subsection{考察}
以上の実験結果をまとめると,提案手法は従来手法よりも元データに近い敵対的サンプルを生成することが示された.

特に成功率および平均距離の指標では,提案手法( $\bm{v}_{\mathrm{raw}}$ )が従来手法を上回る性能を発揮した.図8や実際のデータから示されるように,従来手法では重要度の大きい特徴量に対するノイズの回避傾向から次点の重要度の高い特徴量に対するノイズが大きくなってしまっていた.提案手法はこれらのノイズが抑制され,ノイズの平均距離を小さく,重み距離を同程度にすることができた.

一方で,重み距離の結果において提案手法( $\bm{v}_{\mathrm{sqrt}}$ )が劣る原因としては,重要度算出アルゴリズムにおける平方根操作の影響が挙げられる.平方根操作により重要とそうでない特徴量の差が生まれにくくなり,敵対的サンプルの人間に知覚されづらい性質を表現できなくなったと考えられる.

特徴量重要度について考察すると,提案手法($\bm{v}_{\mathrm{raw}}$)は,従来手法の正規化処理が精度向上を悪化させていたことがわかった.相関係数の絶対値は0から1までにスケーリングされているため,正規化を行う必要がない.この結果,重要度の高い特徴量に対する過剰なノイズの付与を抑制し,より自然な敵対的サンプルを生成することができた.

離散化手法については,平均距離,重み距離ともに四捨五入による離散化が最も優れた性能を示した.一方,ランダム離散化は提案手法についてあまり良い精度を出すことができなかった.これは情報損失がない,もしくは余計な情報に惑わされたためと考えられる.

最後に元データ,従来手法,提案手法($\bm{v}_{\mathrm{raw}}$)による敵対的サンプルを比較した.提案手法は従来手法と比較してより人間に知覚されにくく,元のデータ分布に近い敵対的サンプルを生成できることが確認された.

以上より,提案手法($\bm{v}_{\mathrm{raw}}$)がどの評価においても最も精度が良いことがわかった.