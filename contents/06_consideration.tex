% \subsection{考察}
以上の実験結果をまとめると,提案手法は従来手法よりも元データに近い敵対的サンプルを生成することが示された.

特に成功率および平均距離の指標では,提案手法( $\bm{v}_{\mathrm{raw}}$ )が従来手法を上回る性能を発揮した.図8や実際のデータから示されるように,従来手法では重要度の大きい特徴量に対するノイズの回避傾向から次点の重要度の高い特徴量に対するノイズが大きくなってしまっていたことが言える.提案手法はこれらのノイズが抑制され,ノイズの平均距離を小さく,重み距離を同程度にまですることができた.この結果は,提案手法が重要度算出と離散化の適切な組み合わせにより,必要最小限のノイズで誤分類を誘発する敵対的サンプルを生成できたためと考えられる.

一方で,重み距離の結果において提案手法( $\bm{v}_{\mathrm{sqrt}}$ )が劣る原因としては,重要度算出アルゴリズムにおける平方根操作の影響が挙げられる.平方根は大きい値を小さくするため図8で従来手法と比較するとより滑らかな重要度となっている.分類性能に影響を与える重要な特徴量に対する変化が抑制されなかったことが考えられる.これにより自然な敵対的サンプルの生成を行うことができなかったということがわかる.

さらに,評価結果に基づき元データ,従来手法,提案手法(式(7))による敵対的サンプルを比較した結果,提案手法によるサンプルはより人間に知覚することが難しく,元のデータ分布に近い実運用環境における敵対的サンプル対策として有効であることが確認された.