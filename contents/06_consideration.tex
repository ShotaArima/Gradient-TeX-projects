% \subsection{考察}
以上の実験結果をまとめると,提案手法は従来手法よりも元データに近い敵対的サンプルを生成することが示された.

特に成功率および平均距離の指標では,提案手法( $\bm{v}_{\mathrm{raw}}$ )が従来手法を上回る性能を発揮した.図8や実際のデータから示されるように,従来手法では重要度の大きい特徴量に対するノイズの回避傾向から次点の重要度の高い特徴量に対するノイズが大きくなってしまっていたことが言える.提案手法はこれらのノイズが抑制され,ノイズの平均距離を小さく,重み距離を同程度にまですることができた.この結果は,提案手法が重要度算出と離散化の適切な組み合わせにより,必要最小限のノイズで誤分類を誘発する敵対的サンプルを生成できたためと考えられる.

一方で,重み距離の結果において提案手法( $\bm{v}_{\mathrm{sqrt}}$ )が劣る原因としては,重要度算出アルゴリズムにおける平方根操作の影響が挙げられる.平方根は大きい値を小さくするため図8で従来手法と比較するとより滑らかな重要度となっている.分類性能に影響を与える重要な特徴量に対する変化が抑制されなかったことが考えられる.これにより自然な敵対的サンプルの生成を行うことができなかったということがわかる.

特徴量重要度について考察する,提案手法($\bm{v}_{\mathrm{raw}}$)は,従来手法の重要度算出で行われていた正規化処理が精度向上を悪化させていたことがわかった.すでに相関係数の絶対値というのは,0から1までにスケーリングされているため,正規化を行う必要がないと考えたため,正規化を行わずに重要度を算出している.この結果,重要度の高い特徴量に対して過剰なノイズの付与を抑制することができ,より自然な敵対的サンプルを生成することができた.また,提案手法($\bm{v}_{\mathrm{sqrt}}$)は,平方根の平滑化により重み距離の精度悪化を引き起こしていたことがわかる.これは,平方根操作により重要とそうでない特徴量の差が生まれにくくなり,敵対的サンプルの人間に知覚されづらい性質を表現できなくなってしまったことが考えられる.

離散化手法について考察する.平均距離,重み距離ともに四捨五入による離散化が最も優れた性能を示した.一方,小数点以下の情報損失を懸念して,導入したランダム離散化は,提案手法についてはあまりより精度を出すことはできなかった.これは情報損失がない,もしくは余計な情報に惑わされてしまったということができる.

最後に元データ,従来手法,提案手法($\bm{v}_{\mathrm{raw}}$)による敵対的サンプルを比較した.従来手法と比較し提案手法はより人間に知覚することが難しく,元のデータ分布に近い実運用環境における敵対的サンプル対策として有効であることが確認された.


以上より,提案手法($\bm{v}_{\mathrm{raw}}$)がどの評価において最も精度が良いことがわかった.