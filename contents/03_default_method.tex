

\subsection{LowProFool}
表形式データに対する敵対的サンプルの生成手法として,Balletらが提案したLowProFool\cite{ballet2019imperceptible}がある.表形式データは識別に寄与する各特徴量の重要度が異なることが多い.LowProFoolはこの点に着目し,各特徴量の重要度を元データから算出し,それに基づいたノイズを付加することで誤分類を引き起こすデータを生成するを行う.特徴量の重要度を示すベクトル $\bm{v}$ とノイズベクトル $\bm{r}$ を用い,検出されにくさを表現する.この手法の目的関数は式(1)で定義される.

\begin{equation}
g(r) = L(\bm{x}+\bm{r}, t) + \lambda ||\bm{v} \odot \bm{r}||_2 \tag{2}
\end{equation}

 $L(\bm{x}+\bm{r}, t)$ では,損失関数を定義しており,元データ $\bm{x}$ にノイズ $\bm{r}$ を付加することで意図的に分類させたいラベル $t$ 導くことを表している.$\lambda ||\bm{v} \odot \bm{r}||_2$ では,先ほどのノイズ $\bm{r}$ と特徴量の重要度を表した $\bm{v}$ をアダマール積によって各特徴量ごとに重み付けする.これらの計算結果を元に $\lambda$ は前者のノイズを付加し誤分類へ導くことと、後者の重要度を意識したノイズを付加考慮したノイズの大きさを制約する.この手法では,

元データ $\bm{x}$ にノイズを付加した敵対的サンプル $\bm{x}'$ を式(1)を最小化によって生成する.

\subsection{特徴量の重要度算出方法}

\subsection{従来手法の課題}