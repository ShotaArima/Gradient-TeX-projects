本論文の執筆にあたり,多くの方々から多大なるご支援とご指導を賜りましたことを,心より感謝申し上げます.
まず,指導教員の東京都市大学メディア情報学部情報システム学科 三川健太准教授には研究の最終段階に至るまで,度重なるご指導,ご鞭撻を賜りましたことを深く感謝いたします.
また,東京都市大学大学院環境情報学研究科環境情報学専攻 福田竜也氏には,研究の進行において多くの有益なアドバイスをいただきました.福田氏のご助言により研究をより前に進めることができました.来年以降研究室にいないことが一番心寂しいですが、いろいろ教わったことを元に2年間の院生活を頑張ります。
また、同研究室の中島 悠斗さん、宮川 快士さんにも研究に対する様々なアドバイスを頂きました。中島さんは1年間だけでしたが、研究室での他愛もない雑談がとても楽しく、忘れられません。宮川さんは、来年もお世話になることが多いと思いますが、よろしくお願いします。
さらに,日々のゼミ活動において多くの助言を賜りました。2年間の思い出とともに感謝を申し上げます.

青木 天翔さんは、強化学習についての研究をもくもくとやられていて、ゼミでの発表でいつも新しい視点をもたらしてくれました.

海老根 祥吾さんは、研究の息抜きとしてバドミントンを一緒に楽しむことができました.また昼食に飲んだあら汁は今でもしみてます。

尾崎 祐矢さんは、提出期限ギリギリまで研究を頑張っていました。最後の発表でのスライドは、とてもわかりやすく無事研究が形になってよかったと思いました。

加賀屋 佑樹さんは、4年次にはあまり連絡をとることができませんでしたが、最後に研究を発表し、質疑応答をしている姿をみることができてよかったです。

樫山 宣武さんさんは、2年間のゼミ長という役職を全うし、いつでもメンバーに気をかけてくれました。ときにいじられときにみんなの前でしゃべらされることもありましたが、それも含めて楽しいゼミ活動でした。

志村 光央太さんは、初めてあったときから強烈だった印象があります。飲み会での立ち回りや卒論の時期に能登へのボランティアに行くなど積極的かつ大胆な行動は、私にとって刺激的でした。

新道 舞子さんは、いつも明るく元気で、ゼミ活動を盛り上げてくれました。飲み会の幹事を引き受けてくださりとても助かりました。ゼミ合宿に対して元気に楽しかったといってもらえてとても嬉しかったです。

高野 晃治さんは、愚痴から研究まで様々な話を学生室でしました。社会人になってもストレスはたまり続けていくだろうなと思うので、定期的に飲みに行きましょう。

高橋 暖人さんは、同じく院へ進学すると思います。4年次から会話が多くなり、表に出なかったキャラクターを知ることができ、おもしろかったです。あと2年間また一緒に頑張っていきましょう。

津田 雄理さんは、ゼミ合宿で酔いつぶれている印象が強いです。特に4年次では2日目の夜に叱られるくらい騒ぎ、翌日の運転を三川先生に託すなど驚きの連続でした。社会人では大きな失敗がないことを祈ります。

稗田 天人さんも、同じく院進学をすると思います。この研究室で一番長いときを過ごしてきたと思います。残りの2年間も最後までお互いに頑張りましょう

宮脇 怜央さんは、最初は正直少し近寄りがたい印象がありました。年月を重ね話していくうちにとても優しくて、小さいなことにも気づいてくれるとてもいい人だということに気づきました。こうたをあやしていたりと曲者を扱うのが得意だと思うので、同窓会ではよろしくお願いします。

村井 希優さんは、Slackでレスが早く、気づいていない人に率先して連絡を回したりなど、縁の下の力持ちとしてとても頼りになりました。また、研究室の雰囲気を和ませるなど、なくてはならない人でした。気苦労も多かったと思いますが、お疲れ様でした。

佐竹 航希さんには、研究の環境構築において多大なるご協力をいただきました。自分自身の力不足な局面でいつも助けになっていたのは、いつもあなたです。この研究室に途中から参加したと思いますが、この研究室を選んでくれてありがとうと心のそこから感謝申し上げます.

皆様のご支援とご指導に,改めて感謝の意を表します.