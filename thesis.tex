\documentclass[12pt,uplatex]{jsarticle}   % 日本語
\usepackage{tcu-thesis}
\usepackage{fancyhdr}
\usepackage{bm} % 太字の数式
\usepackage{amsmath} % 数式

% オリジナルマクロ by Kyohei Fushida and Kenji Fujiwara
\newcommand{\li}{\item}
\newcommand{\ol}[1]{\begin{enumerate}#1\end{enumerate}}
\newcommand{\ul}[1]{\begin{itemize}#1\end{itemize}}
\newcommand{\dl}[1]{\begin{description}#1\end{description}}
\newcommand{\equ}[1]{\begin{equation}#1\end{equation}}
\newcommand{\eqenum}[1]{\begin{align}#1\end{align}}
\newcommand{\eqa}[1]{\begin{align}#1\end{align}}
\newcommand{\eqmul}[1]{\begin{equation}\begin{split}#1\end{split}\end{equation}}
\usepackage{xcolor}

%\newcommand{\figref}[1]{図\ref{#1}}
%\newcommand{\tabref}[1]{表\ref{#1}}
\renewcommand{\subsecref}[1]{\ref{#1}項}
\newcommand{\chapref}[1]{\ref{#1}章}

\definecolor{darkgreen}{rgb}{0, 0.5, 0} %下の色指定で使う
\definecolor{whitesmoke}{rgb}{0.99, 0.99, 0.99} %下の色指定で使う
\newcommand{\red}{\color{red}}%red
\newcommand{\blue}{\color{blue}}
\newcommand{\green}{\color{teal}}
\newcommand{\purple}{\color{violet}}%violet
\newcommand{\black}{\color{black}}
\newcommand{\ea}{\xspace\emph{et al.}\xspace}
\newcommand{\TODO}[1]{\nb{TODO:}{#1}}


\newcommand{\methods}{\#methods}
\newcommand{\colorrow}{\rowcolor[rgb]{0.851,0.851,0.851}}
\newcommand{\colorcel}{\cellcolor[rgb]{0.851,0.851,0.851}}
\def\et{\xspace et\ al.\xspace}
\def\ie{i.e.,\xspace}
\def\eg{e.g.,\xspace}
\newcommand{\nb}[2]{
    \fcolorbox{gray}{yellow}{\bfseries\sffamily\scriptsize#1}
    {\sf\small\textit{\textcolor{blue}{#2}}}
   }

\newtheorem{defi}{定義}
\newtheorem{hypo}{仮説}

% 必要なパッケージの読み込み
% とりあえず最低限使っておいて欲しいパッケージのみ書いています.
% 色々なパッケージがあります.藤原が過去に使ってたパッケージなどは博士論文のファイルを参照してください.
\usepackage[dvipdfmx]{graphicx}
\usepackage{cite} % 参考文献を [3,4]のようにまとめるのに便利.
\usepackage{url} % urlを記載するときに使う.

% 参考文献を引用された順で出力する
\bibliographystyle{junsrt}

% 学籍番号
\studentnumber{2172010}

% 日本語題目
\title{入力データの構造を考慮したLowProFoolアルゴリズム\\による敵対的サンプルの生成に関する研究}

% 英語題目
\etitle{Generation of Adversarial Samples Using Modified LowProFool Algorithm with Consideration of Input Data Structure}

% 日本語氏名(姓と名の間に空白(半角)を入れて下さい)
\author{有馬 祥太}

% 欧文氏名(first name, last name の順に記入し、先頭文字のみを大文字にする。)
\eauthor{Shota Arima}

% 専攻
\department{\media}
\course{\jsys}

% 論文提出年月日
\syear{2025}
\smonth{1}
\sday{30}

% キーワード5〜6個 (in LaTeX)
\keywords{敵対的サンプル,表形式データ,LowProFoolアルゴリズム,データ離散化,特徴量重要度}

% 5 or 6 Keywords (in LaTeX)
\ekeywords{Adversarial Samples, Table Data, LowProFool Algorithm, Data discretization, Importance of Features}

% 内容梗概 (in LaTeX)
%   Abstractは必須です。
%   サンプルほど長々と書く必要はありません.
%   注: 行の先頭が\\で始まらないようにすること。
\abstract{%
この研究では,機械学習モデルに対する敵対的サンプル生成において,表形式データのおける敵対的サンプルの特性を考慮した改良手法を提案しています。
従来のLowProFoolアルゴリズムでは特定の特徴量に対して過度にノイズが付与されてしまう可能性と離散値を持つ表形式データに対して連続値の敵対的サンプルを生成してしまう課題がありました。
そこで本研究では,特徴量の重要度算出方法を改善し,出力データの離散化を行うことで,より自然な敵対的サンプルの生成を実現しました。
実験結果から,提案手法により過度なノイズの付与を抑制でき,より現実的な敵対的サンプルの生成が可能になりました。
}

%%%%%%%%%%%%%%%%%%%%%%%%% document starts here %%%%%%%%%%%%%%%%%%%%%%%%%%%%

\begin{document}
%
% 表紙 および アブストラクト
%
\titlepage
\jabstractpage
%
% 目次
%
\tableofcontents
\newpage

% 以降本文
\pagenumbering{arabic} % ページ番号をアラビア数字に戻す

% この章構成はあくまで一例です.
% \section:章
% \subsection: 節
% \subsubsection: 項
% として使ってください.
% 研究の内容によって自分が適切だと思う章構成を考えて下さい.
% もちろん構成について藤原と相談するのも良いと思います.
% \section{テンプレート}
% %サンプルです.必要なければファイルごと削除してください.
% 「はじめに」では研究の背景を述べつつ研究の目的を述べるのが一般的です.
% 仰々しい論文の場合は論文の構成(以降の章構成)を説明したりもします.

% 最初は世の中(といっても情報系分野ぐらいで十分ですが)全体の背景を述べつつ,
% 徐々に的を絞ってどうして自身の研究を行う必要があるのかを上手く説明できると素晴らしいです.
% 研究の背景となる課題や既存研究を紹介する際には必ず参考文献を引用するようにしましょう.

% 「はじめに」は他の章に比べて非常に作成が難しいです.
% どういう流れで書けばいいのか困った際は指導教員に聞くのもよいと思いまし,とりあえず書ける章(自分が実際に手を動かした章)から書いていくのも良いと思います.


% \subsection{論文を書く際の注意事項}
% \begin{itemize}
%     \item 句読点は全角のカンマとピリオドである「,」と「.」を使って下さい.卒論執筆期間中は一時的にIMEの設定を変更して常にカンマとピリオドが使用されるようにしておくと安全だと思います.
%     \item 表記揺れに注意しましょう.
%     \item ですます調はNGです.
%     \item 未定義語は使わない.一般的でない用語が出てくるタイミングで必ず説明を入れる.
% \end{itemize}


\section{序論}
\subsection{研究背景}
近年,情報技術の進歩に伴い,機械学習モデルが多くの分野で活用され始めている.金融,医療,交通などの領域において,膨大なデータを解析することで効率的かつ高度な意思決定が可能となり,私たちの生活は利便性と効率性を向上させる多くの恩恵を受けられるようになった.一方で,これらの機械学習モデルに依存する場面が増えるにつれて,機械学習モデルの誤作動や悪用が引き起こすリスクも拡大している。例えば,ば,医療分野における誤診や,交通分野における自動運転車の誤作動が発生した場合,人的被害や社会的混乱を招く可能性がある.そのため、機械学習モデルがもたらす利便性を最大化しつつ、それらが安全かつ信頼できるものであることを保証することが極めて重要である.こうした状況の中で特に注目されている課題が、機械学習モデルの脆弱性を悪用した攻撃手法である。その一例として、敵対的サンプルによる攻撃が挙げられる。敵対的サンプルとは,人間には見分けのつかない微細なノイズをデータに付与することで機械学習モデルの誤分類を引き起こすデータである.このようなデータを用いた攻撃によって機械学習モデルが重大な誤りを犯す可能性がある.例えば,銀行のローンの審査システムについての機械学習モデルを想定すると,本来なら認可すべきではないリスクの高いローンを誤って承認されるケースが考えられる.このような問題に対処するため,機械学習モデルの安全性が求められており,よりノイズの小さい敵対的サンプルの生成によってよりロバストな機械学習モデルの構築の貢献し,安全性を向上に寄与すると考えられる.

敵対的サンプルは主に画像データを対象に研究が行われてきた.特に自動運転における物体検知の研究においてこの敵対的サンプルによる誤分類問題がよく取り上げられていた.しかし画像データだけではなく,表形式データに対する攻撃もまた無視できない課題である.表形式データは,金融,医療,ビジネス領域で広く利用されており,機械学習モデルの入力データとしても多く用いられている.画像用の敵対的サンプル生成手法を適用すると非現実的なサンプルが生成される可能性がある.そのため,表形式データの敵対的サンプルの生成手法では,入力データの特性に合わせた敵対的サンプルの生成手法が求められる.

\subsection{研究目的}
上記の議論のもと本研究では,表形式データの特徴を考慮したより自然な敵対的サンプルを生成する手法を提案することを目的とする.元データの分布に合わせてより少ないノイズを加え,連続値として出力される非現実的なサンプルの問題を解決する.以上より,より自然な表形式データに対する敵対的サンプルの生成が可能となる.提案手法により,本研究の目的が達成されることで,既存の機械学習モデルの弱点をより正確に把握することが可能となる.これにより,モデルの脆弱性を評価し,敵対的サンプルによる誤分類を防ぐための対応策を事前に講じることができ,モデルの安全性向上に貢献することができる.

\section{準備}
%サンプルです.必要なければファイルごと削除してください.
この章では自分の研究に関連する既存研究,自身の研究の説明に必要となりそうな技術について紹介してください.
例えば,micro:bitを扱う研究などではmicro:bitとはそもそも何なのかを説明する節を追加しても良いです.
卒論等では,そもそも自身の研究に関する背景知識がそれほど無い人でも何をしているのかが分かるように,背景となる研究・技術をしっかり自身の言葉でまとめてください.
ここでWeb等に誰かが書いた内容を丸々コピーしていた場合は,最悪卒論として受理しないという可能性もあるので気をつけてください.
ある程度subsectionを作成して,構造化できると良いと思います.

\subsection{敵対的サンプル}

\subsection{LowProFool}

\subsection{特徴量の重要度算出方法}

\subsection{従来手法の課題}

\section{提案手法}
%サンプルです.必要なければファイルごと削除してください.
この章では提案手法について説明します.
図や表をうまく使って,誰もが理解しやすい記述をしてください.

\subsection{重要度算出法の改善}

\subsection{出力データの離散化手法}


\section{実験}
%サンプルです.必要なければファイルごと削除してください.
この章では,提案手法を実際に適用する実験について説明してください.
手法の細かいパラメータ設定などは提案手法などではなく実験の章で説明することが多いです.
あくまで,提案手法はより広い範囲に手法を適用できるような抽象的な手順を書くのに留めましょう.
\subsection{実験条件}

\subsection{評価指標}

\subsection{実験結果と考察}

\subsection{提案手法の有効性}

\section{結論}
%サンプルです.必要なければファイルごと削除してください.
この章では卒論全体を振り返って論文をまとめてください.
また,今後の課題等があればそれについて説明してもらっても構いません.
\subsection{まとめ}

\subsection{今後の課題}


% 謝辞
\acknowledgements
% 謝辞の文面は過去の(豊田高専を含む)謝辞や,藤原の博士論文を参考にしてください.
% 博士論文の謝辞は長大ですが,卒業研究論文の謝辞をそんなに長くする必要は全くないと思います.
本論文の執筆にあたり、多くの方々から多大なるご支援とご指導を賜りましたことを,心より感謝申し上げます.
まず,指導教員の東京都市大学メディア情報学部情報システム学科 三川健太准教授には研究の最終段階に至るまで,度重なるご指導,ご鞭撻を賜りましたことを深く感謝いたします.
また,東京都市大学大学院環境情報学研究科環境情報学専攻 福田竜也氏には,研究の進行において多くの有益なアドバイスをいただきました.福田氏のご助言により研究をより前に進めることができました.心から感謝いたします.
さらに,日々のゼミ活動において多くの助言を賜りました,三川研究室の皆様にも感謝いたします.特に研究の環境構築において多大なるご協力をいただいた同研究室の佐竹航希さんに,心より感謝いたします.
皆様のご支援とご指導に,改めて感謝の意を表します.

% 参考文献
% BibTeX を使って参考文献のリストをつくって下さい.
% proceedings.bib: 雑誌名や国際会議の名称をマクロ化しています
% ref.bib: ここに参考文献を書くと良いと思います.元々博士論文に使った論文の情報が載っています.
% ref.bib以外のファイルに分けることも可能です.単純にカンマで区切ってファイル名(.bib除く)を
% \bibliographyの引数に追加してください.
\bibliography{proceedings,ref}

% 付録
% 付録には,本文に直接載せるべきではない,文字通り付録となる要素を載せます.
% 例えば,詳細な数式の展開であったり,アンケートや実験の詳細な(ほぼ完全な)結果だったりを載せます.
% 必要であれば以下の3行のコメントアウトを外して付録を加えてください.
\appendix
% サンプルです.全部削除してください.


\end{document}

