\documentclass[12pt,uplatex]{jsarticle}   % 日本語
\usepackage{tcu-thesis}
\usepackage{fancyhdr}
\usepackage{bm} % 太字の数式
\usepackage{amssymb} % 数式
\usepackage{amsmath} % 数式
\usepackage{float} % 図表の位置
\usepackage{listings} % ソースコード
\usepackage{color} % ソースコードの色付け

% ソースコードのスタイル設定
\lstset{
  basicstyle=\ttfamily\footnotesize,
  numbers=left,
  numberstyle=\tiny\color{gray},
  stepnumber=1,
  numbersep=5pt,
  backgroundcolor=\color{white},
  showspaces=false,
  showstringspaces=false,
  showtabs=false,
  frame=single,
  rulecolor=\color{black},
  tabsize=2,
  captionpos=b,
  breaklines=true,
  breakatwhitespace=false,
  title=\lstname,
  keywordstyle=\color{blue},
  commentstyle=\color{green},
  stringstyle=\color{red},
  escapeinside={\%*}{*)},
  morekeywords={*,...}
}

% Python用のスタイル設定
\lstdefinestyle{python}{
  language=Python,
  keywordstyle=\color{blue},
  commentstyle=\color[RGB]{0,100,0},     %DarkGreen
  stringstyle=\color[RGB]{128,0,0},      %Maroon
  frame=single,
  breaklines=true,
  basicstyle=\ttfamily\footnotesize
}

% Jupyter Notebook用のスタイル設定
\lstdefinestyle{jupyter}{
  language=Python,
  keywordstyle=\color{blue},
  commentstyle=\color[RGB]{0,100,0},     %DarkGreen
  stringstyle=\color[RGB]{128,0,0},      %Maroon
  frame=single,
  breaklines=true,
  basicstyle=\ttfamily\footnotesize,
  morekeywords={In, Out, print}
}


% アルゴリズムのレイアウト
\newenvironment{algorithm_step}
{\begin{list}{}
    {\setlength{\leftmargin}{6em}  % Step X)の部分の幅
     \setlength{\itemindent}{0em}
     \setlength{\labelsep}{0em}
     \setlength{\labelwidth}{6em}  % Step X)の部分の幅
     \setlength{\itemsep}{1em}     % ステップ間の縦方向の間隔
     \renewcommand{\makelabel}[1]{##1\hfill}}}
{\end{list}}

% 数式カウンターの定義
\newcounter{myeqcounter} % 独自カウンターを定義

\newcommand{\autoequation}[1]{
\refstepcounter{myeqcounter} % カウンターを進める
\begin{equation}
#1 \tag{\themyeqcounter} % カウンター値をタグに適用
\end{equation}
}


% オリジナルマクロ by Kyohei Fushida and Kenji Fujiwara
\newcommand{\li}{\item}
\newcommand{\ol}[1]{\begin{enumerate}#1\end{enumerate}}
\newcommand{\ul}[1]{\begin{itemize}#1\end{itemize}}
\newcommand{\dl}[1]{\begin{description}#1\end{description}}
\newcommand{\equ}[1]{\begin{equation}#1\end{equation}}
\newcommand{\eqenum}[1]{\begin{align}#1\end{align}}
\newcommand{\eqa}[1]{\begin{align}#1\end{align}}
\newcommand{\eqmul}[1]{\begin{equation}\begin{split}#1\end{split}\end{equation}}
\usepackage{xcolor}

%\newcommand{\figref}[1]{図\ref{#1}}
%\newcommand{\tabref}[1]{表\ref{#1}}
\renewcommand{\subsecref}[1]{\ref{#1}項}
\newcommand{\chapref}[1]{\ref{#1}章}

\definecolor{darkgreen}{rgb}{0, 0.5, 0} %下の色指定で使う
\definecolor{whitesmoke}{rgb}{0.99, 0.99, 0.99} %下の色指定で使う
\newcommand{\red}{\color{red}}%red
\newcommand{\blue}{\color{blue}}
\newcommand{\green}{\color{teal}}
\newcommand{\purple}{\color{violet}}%violet
\newcommand{\black}{\color{black}}
\newcommand{\ea}{\xspace\emph{et al.}\xspace}
\newcommand{\TODO}[1]{\nb{TODO:}{#1}}


\newcommand{\methods}{\#methods}
\newcommand{\colorrow}{\rowcolor[rgb]{0.851,0.851,0.851}}
\newcommand{\colorcel}{\cellcolor[rgb]{0.851,0.851,0.851}}
\def\et{\xspace et\ al.\xspace}
\def\ie{i.e.,\xspace}
\def\eg{e.g.,\xspace}
\newcommand{\nb}[2]{
    \fcolorbox{gray}{yellow}{\bfseries\sffamily\scriptsize#1}
    {\sf\small\textit{\textcolor{blue}{#2}}}
   }

\newtheorem{defi}{定義}
\newtheorem{hypo}{仮説}

% 必要なパッケージの読み込み
% とりあえず最低限使っておいて欲しいパッケージのみ書いています.
% 色々なパッケージがあります.藤原が過去に使ってたパッケージなどは博士論文のファイルを参照してください.
\usepackage[dvipdfmx]{graphicx}
\usepackage{cite} % 参考文献を [3,4]のようにまとめるのに便利.
\usepackage{url} % urlを記載するときに使う.

% 参考文献を引用された順で出力する
\bibliographystyle{junsrt}

% 学籍番号
\studentnumber{2172010}

% 日本語題目
\title{入力データの構造を考慮したLowProFoolアルゴリズム\\による敵対的サンプルの生成に関する研究}

% 英語題目
\etitle{Generation of Adversarial Samples Using Modified LowProFool Algorithm with Consideration of Input Data Structure}

% 日本語氏名(姓と名の間に空白(半角)を入れて下さい)
\author{有馬 祥太}

% 欧文氏名(first name, last name の順に記入し,先頭文字のみを大文字にする。)
\eauthor{Shota Arima}

% 専攻
\department{\media}
\course{\jsys}

% 論文提出年月日
\syear{2025}
\smonth{1}
\sday{30}

% % キーワード5〜6個 (in LaTeX)
% \keywords{敵対的サンプル,表形式データ,LowProFoolアルゴリズム,データ離散化,特徴量重要度}

% % 5 or 6 Keywords (in LaTeX)
% \ekeywords{Adversarial Samples, Table Data, LowProFool Algorithm, Data discretization, Importance of Features}

% 内容梗概 (in LaTeX)
%   Abstractは必須です。
%   サンプルほど長々と書く必要はありません.
%   注: 行の先頭が\\で始まらないようにすること.
% \abstract{%
% この研究では,機械学習モデルに対する敵対的サンプル生成において,表形式データのおける敵対的サンプルの特性を考慮した改良手法を提案しています.
% 従来のLowProFoolアルゴリズムでは特定の特徴量に対して過度にノイズが付与されてしまう可能性と離散値を持つ表形式データに対して連続値の敵対的サンプルを生成してしまう課題がありました.
% そこで本研究では,特徴量の重要度算出方法を改善し,出力データの離散化を行うことで,より自然な敵対的サンプルの生成を実現しました.
% 実験結果から,提案手法により過度なノイズの付与を抑制でき,より現実的な敵対的サンプルの生成が可能になりました.
% }

%%%%%%%%%%%%%%%%%%%%%%%%% document starts here %%%%%%%%%%%%%%%%%%%%%%%%%%%%

\begin{document}
%
% 表紙 および アブストラクト
%
\titlepage
% \jabstractpage
%
% 目次
%
\tableofcontents
\newpage

% 以降本文
\pagenumbering{arabic} % ページ番号をアラビア数字に戻す

% この章構成はあくまで一例です.
% \section:章
% \subsection: 節
% \subsubsection: 項
% として使ってください.
% 研究の内容によって自分が適切だと思う章構成を考えて下さい.
% もちろん構成について藤原と相談するのも良いと思います.
% \section{テンプレート}
% %サンプルです.必要なければファイルごと削除してください.
% 「はじめに」では研究の背景を述べつつ研究の目的を述べるのが一般的です.
% 仰々しい論文の場合は論文の構成(以降の章構成)を説明したりもします.

% 最初は世の中(といっても情報系分野ぐらいで十分ですが)全体の背景を述べつつ,
% 徐々に的を絞ってどうして自身の研究を行う必要があるのかを上手く説明できると素晴らしいです.
% 研究の背景となる課題や既存研究を紹介する際には必ず参考文献を引用するようにしましょう.

% 「はじめに」は他の章に比べて非常に作成が難しいです.
% どういう流れで書けばいいのか困った際は指導教員に聞くのもよいと思いまし,とりあえず書ける章(自分が実際に手を動かした章)から書いていくのも良いと思います.


% \subsection{論文を書く際の注意事項}
% \begin{itemize}
%     \item 句読点は全角のカンマとピリオドである「,」と「.」を使って下さい.卒論執筆期間中は一時的にIMEの設定を変更して常にカンマとピリオドが使用されるようにしておくと安全だと思います.
%     \item 表記揺れに注意しましょう.
%     \item ですます調はNGです.
%     \item 未定義語は使わない.一般的でない用語が出てくるタイミングで必ず説明を入れる.
% \end{itemize}


\section{序論}
\subsection{研究背景}
近年,情報技術の進歩に伴い,機械学習モデルが多くの分野で活用され始めている.金融,医療,交通などの領域において,膨大なデータを解析することで効率的かつ高度な意思決定が可能となり,私たちの生活は利便性と効率性を向上させる多くの恩恵を受けられるようになった.一方で,これらのモデルに依存する場面が増えるにつれて,機械学習モデルの誤動作や悪用が引き起こすリスクも拡大している.

特に,機械学習モデルの脆弱性を悪用した攻撃手法が注目されており,その中でも敵対的サンプルによる攻撃が重要な課題となっている.\cite{MBSD-AdversarialTraining}
敵対的サンプルとは,人間には見分けのつかない微細なノイズをデータに付与することで,機械学習モデルの誤分類を引き起こすデータである.\cite{MBSD-AdversarialExample}このようなデータによる攻撃により,モデルが誤った結果を出力することで,安全性や信頼性が損なわれるだけでなく,重大な社会的・経済的影響を引き起こす可能性がある.たとえば,自動運転車の物体認識システムに敵対的サンプルを送り込むことで,信号や歩行者を誤認し,交通事故につながるリスクがある.\cite{AdversarialMachineLearning:BayesianPerspectives}

さらに,金融や医療の分野においても同様のリスクが存在する.例えば,銀行のローンの審査システムについての機械学習モデルを想定すると,本来なら認可すべきではないリスクの高いローンを誤って承認されるケースが考えられる.このように,敵対的サンプルはあらゆる分野で機械学習モデルの信頼性を脅かす要因となっており,これに対処するための安全性向上が求められている.

一方で,これらのモデルに依存する場面が増えるにつれて,機械学習モデルの誤動作や悪用が引き起こすリスクも拡大している.特に,機械学習モデルの脆弱性を悪用した攻撃手法が注目されており,その中でも敵対的サンプルによる攻撃が重要な課題となっている.
敵対的サンプルとは,人間には見分けのつかない微細なノイズをデータに付与することで,機械学習モデルの誤分類を引き起こすデータである.このような攻撃により,モデルが誤った結果を出力することで,安全性や信頼性が損なわれるだけでなく,重大な社会的・経済的影響を引き起こす可能性がある.例えば,自動運転車の物体認識システムに敵対的サンプルを送り込むことで,信号や歩行者を誤認し,交通事故につながるリスクがある.さらに,金融や医療の分野においても同様のリスクが存在する.例えば,銀行のローンの審査システムについての機械学習モデルを想定すると,本来なら認可すべきではないリスクの高いローンを誤って承認されるケースが考えられる.このように,敵対的サンプルはあらゆる分野で機械学習モデルの信頼性を脅かす要因となっており,これに対処するための安全性向上が求められている.

このような問題に対処するため,機械学習モデルの安全性が求められており,よりノイズの小さい敵対的サンプルの生成によってよりロバストな機械学習モデルの構築の貢献し,安全性を向上に寄与すると考えられる.

敵対的サンプルは主に画像データを対象に研究が行われてきた.特に自動運転における物体検知の研究\cite{MBSD-automobile}においてこの敵対的サンプルによる誤分類問題がよく取り上げられていた.しかし画像データだけではなく,表形式データに対する攻撃もまた無視できない課題である.表形式データは,金融,医療,ビジネス領域で広く利用されており,機械学習モデルの入力データとしても多く用いられている.画像用の敵対的サンプル生成手法を適用すると非現実的なサンプルが生成される可能性がある.そのため,表形式データの敵対的サンプルの生成手法では,入力データの特性に合わせた敵対的サンプルの生成手法が求められる.

\subsection{研究目的}
上記の議論のもと本研究では,表形式データの特徴を考慮したより自然な敵対的サンプルを生成する手法を提案することを目的とする.元データの分布に合わせてより少ないノイズを加え,連続値として出力される非現実的なサンプルの問題を解決する.以上より,より自然な表形式データに対する敵対的サンプルの生成が可能となる.提案手法により,本研究の目的が達成されることで,既存の機械学習モデルの弱点をより正確に把握することが可能となる.これにより,モデルの脆弱性を評価し,敵対的サンプルによる誤分類を防ぐための対応策を事前に講じることができ,モデルの安全性向上に貢献することができる.

\section{準備}
% \subsection{問題設定}
% 本研究では,銀行のローンの審査システムについての機械学習モデルを想定し,認可拒否について誤分類を引き起こす敵対的サンプルを生成することを考える.銀行のローンを申請する顧客情報と正解データをもとに機械学習モデルが学習し,テストデータに対する分類結果を出力する.敵対的サンプルによる誤分類は,銀行にとってリスクの高いローンを誤って認可してしまう可能性がある.このような問題に対処するため,機械学習モデルの安全性が求められており,敵対的サンプルに対する防御手法が必要である.また同時に,モデルの脆弱性を理解することは,安全性を向上させるために重要である.

\subsection{敵対的学習}
機械学習モデルは、膨大なデータからパターンを学習し、予測や分類を行う。しかし、そのデータに微小な変更(敵対的サンプル)が加えられることで、人間には明らかに正しいと認識されるデータを、モデルが誤分類してしまうケースがある。これにより、安全性が求められる分野(例:顔認証や金融審査など)で深刻な問題が発生する可能性があります。

敵対的学習は,機械学習モデルの堅牢性を向上させるための手法として誕生した.従来の機械学習モデルは,訓練データに対して高い精度を示す一方で,敵対的サンプルと呼ばれる微細ななノイズを含むデータに対しては脆弱である。
ことが知られている.これにより,モデルが誤分類を引き起こし,セキュリティ上のリスクが生じる可能性がある.

goodfellowらの研究\cite{goodfellow2015explaining}でこの敵対的学習について提案された.この手法では,正常データと敵対的サンプルの特徴をAIに学習させる防御手法である.機械学習モデルの学習時において,正常データと敵対的サンプルに対する誤差(Loss)をそれぞれ計算し,これらを足し合わせた値を基のモデルの重み $\bm{w}$ を更新することで,敵対的サンプルの特徴を学習する.

以下に実際の敵対的学習の流れを示す.\cite{MBSD-AdversarialTraining}

\begin{enumerate}

    \item 学習中の機械学習モデルを利用して敵対的サンプルを作成する

    下に示す図1は、敵対的学習の最初のステップである、元データから敵対的サンプルを作成するプロセスを示している。この図では、元画像に微細なノイズを加えることで、モデルが誤分類を引き起こす敵対的サンプルが生成される様子を視覚的に表している。最初に元データの画像を機械学習モデルに入力し,予測ラベル $y$ を得る.次に,この予測と正解データを比較した誤差を $Loss$ 関数で計算し,この誤差を最小化するように敵対的サンプルを生成している.
    
    \begin{figure}[H]
        \centering
        \includegraphics[width=0.8\textwidth]{images/敵対的学習1.png}
        \caption{敵対的学習1:元データによる}
        \label{fig:adversarial_learning1}
    \end{figure}
    
    \item 機械モデルに正常なデータ $\bm{x}$ と敵対的サンプル $\bm{x}'$ を入力し,それぞれの誤差 $Loss$ を得る

    図2の上の画像が元データで下の画像が先ほどノイズを加えた画像である.それぞれ機械学習モデルに入力し,それぞれの誤差を取得する.

    \begin{figure}[H]
        \centering
        \includegraphics[width=0.8\textwidth]{images/敵対的学習2.png}
        \caption{敵対的学習2:元データと敵対敵サンプルの誤差を取得する}
        \label{fig:adversarial_learning2}
    \end{figure}

    \item それぞれ得た誤差 $Loss(x, y), Loss(x', y)$ に重み係数 $\alpha$ をつけて足し合わせる

    元データと敵対的サンプル両方の誤差を取得し,重み $\alpha$ をつけて足し合わせることで,敵対的サンプルの挙動のバランスを以下の式(1)で調整できる.
    \autoequation{\alpha \cdot Loss(x,y)+(1−\alpha) \cdot Loss(\tilde{x},y)}
    \begin{figure}[H]
        \centering
        \includegraphics[width=0.8\textwidth]{images/敵対的学習3.png}
        \caption{敵対的学習3:元データと敵対敵サンプルの誤差を足し合わせる}
        \label{fig:adversarial_learning3}
    \end{figure}

    \item 足し合わされた誤差 $Loss$ が最小となるように,重み $\bm{w}$ を更新する

    図1で学習された機械学習モデルの重み $w$ を敵対的サンプルに対応させるため更新する.
    先ほどの式(1)を最小化することで,重み $w$ を $w_{\_new}$ に更新する.これにより,敵対的サンプルの挙動を抑えることができた機械学習モデルを作成することができる.
    
    \begin{figure}[H]
        \centering
        \includegraphics[width=0.8\textwidth]{images/敵対的学習4.png}
        \caption{敵対的学習4:足し合わされた誤差が最小となるよう重みを更新する}
        \label{fig:adversarial_learning4}
    \end{figure}

\end{enumerate}

以上の流れにより,敵対的学習を行うことで,機械学習モデル敵対的サンプルの挙動に対応することができ,堅牢性を向上させることができる.

\subsection{敵対的サンプル}
先ほどの敵対的学習の説明の中の一番最初のステップで生成した敵対的サンプルについて説明する.改めて敵対的サンプルとは,人間には見分けのつかない微細なノイズをデータに付与することで機械学習モデルの誤分類を引き起こすデータである.\cite{MBSD-AdversarialExample}運用している機械学習モデルを狙った攻撃に使用されることがある.

敵対的サンプルによる誤分類を確認した実験をGoodfellowらが行っている\cite{goodfellow2015explaining}.
\begin{figure}[H]
    \centering
    \includegraphics[width=0.8\textwidth]{images/goodfellow_panda.png}
    \caption{敵対的サンプルの例:パンダの敵対的サンプル画像によってテナガザルと誤分類する\cite{goodfellow2015explaining}}
    \label{fig:adversarial_example}
\end{figure}

この研究では,様々な動物に関する画像を学習した機械学習モデルに対して,人間には見分けがつかない微細なノイズを付与したパンダの画像を入力するとテナガザルと誤分類されることを示した.

このような画像データに対する敵対的サンプルの生成は行うことができる.しかし,表形式データに対する敵対的サンプルは,先ほどの手法を用いて生成することは難しい.その理由として,表形式データの特徴量は画像データと異なる性質を持つためである.画像データの場合,各ピクセルの値は0から255の範囲の連続値として扱うことができ,わずかな変化は人間の目では認識できないことが多い.一方で,表形式データの特徴量には,年齢や収入といった数値データだけでなく,性別や職業といったカテゴリカル変数も含まれる.また,各特徴量は独立した意味を持っており,それぞれの特徴量の変化は明確な意味の変化を伴う.例えば,先ほどのパンダの画像の場合,個々のピクセル値にわずかな変化を加えても,人間にとってはそれが「パンダの画像」であることに変わりはない.しかし,ローン申請データにおいて,年収を示す特徴量に対してわずかな変化を加えた場合,その変化は申請者の経済状況を直接的に変えてしまう可能性がある.さらに,職業などのカテゴリカルデータの場合,わずかな変化という概念自体が適用できない.このような表形式データの特性により,画像データに対する敵対的サンプル生成手法をそのまま適用することは適切ではない.そこで,表形式データの特性を考慮した新たな敵対的サンプル生成手法が必要となる.特に,各特徴量の重要度や,特徴量の種類(連続値かカテゴリカル値か)を考慮した手法が求められる.


\subsection{使用するデータセット}
今回使用するデータは,OpenMLが提供しているドイツにおける信用情報データセットcredit-gである.このデータセットは,ドイツの銀行の顧客情報を元に,ローンの信用リスクを予測するためのデータセットである.このデータセットで使用した特徴量は,以下の通りである.

\begin{table}[H]
    \centering
    \caption{信用情報データセットの特徴量}
    \begin{tabular}{|l|l|l|}
        \hline
        特徴量名 & 意味 & 種類 \\ \hline
        checking\_status & 既存の当座預金口座のステータス(ドイツマルク) & カテゴリ \\ \hline
        duration & 期間(月) & 数値 \\ \hline
        credit\_amount & クレジットの金額(申請する与信金額) & 数値 \\ \hline
        savings\_status & 貯蓄口座/債券のステータス(ドイツマルク建て) & カテゴリ \\ \hline
        employment & 現在の雇用年数 & カテゴリ \\ \hline
        installment\_commitment & 可処分所得に対する分割払いの割合 & 数値 \\ \hline
        residence\_since & X年からの現在の居住地 & 数値 \\ \hline
        age & 年齢 & 数値 \\ \hline
        existing\_credits & この銀行の既存クレジット数 & 数値 \\ \hline
        num\_dependents & 扶養家族数 & 数値 \\ \hline
        own\_telephone & 電話 & カテゴリ \\ \hline
        foreign\_worker & 外国人労働者 & カテゴリ \\ \hline
        target & ローン承認 (true) or ローン却下(false) & カテゴリ \\ \hline
    \end{tabular}
    \label{tab:credit_g_features}
\end{table}

カテゴリ変数についてはダミー変数化し離散値として扱う.また数値の特徴量についても全て離散値であることがわかった.このデータセットは,ローンの信用リスクを予測するためのデータセットであるため,ローンの承認結果を予測する二値分類問題として扱う.

\section{従来手法}
\subsection{LowProFool} % 概要
表形式データに対する敵対的サンプルの生成手法として,Balletらが提案したLowProFool\cite{ballet2019imperceptible}がある.表形式データの特徴として識別に寄与する各特徴量の重要度が異なることが多い.LowProFoolはこの点に着目し各特徴量の重要度を元データから算出し,それに基づいたノイズを付加することで誤分類を引き起こし,検知されにくい敵対的サンプルを生成する.

% LowProFoolの問題設定
LowProFoolは検知されにくい敵対的サンプルを生成するため,以下の最適化問題を解く.
\autoequation{\bm{r}^* = \mathrm{arg}_{\bm{r}} \min d(\bm{r}) \text{ for } \bm{r} \in \mathbb{R}^D}
\autoequation{\text{s.t. }  f(\bm{x}) = s \neq f(\bm{x}+\bm{r}^*) = t,  \bm{x}+\bm{r}^* \in A}

式(2)はノイズベクトル $\bm{r}$ を最小化することを式(3)は最小化に必要な制約を表している.
式(2)の中で用いられる $d(\bm{r})$ はノイズの大きさを評価する指標で式(4)のように定義される.
\autoequation{d_{\bm{v}}(r) = ||\bm{r} \odot \bm{v}||^2_p}
ここで $\bm{v}$ は各特徴量の重要度を表す特徴量重要度ベクトルであり,ノイズベクトル $\bm{r}$ とのアダマール積 $\odot$ を用い,その大きさを評価している.評価指標は $\ell_2$ ノルムを用いている.また,ノイズベクトル $\bm{r}$ は元データと同じ $D$ 次元の実数空間から選ばれる.これにより,重要度が高い特徴量には小さなノイズが適用され,重要度が低い特徴量には比較的大きなノイズが許容される.
式(3)の一つ目の制約 $f(\bm{x}) = s \neq f(\bm{x}+\bm{r}^*) = t$ は,元データ $\bm{x}$ の分類結果 $s$ とノイズを付加したデータ $\bm{x}+\bm{r}^*$ の分類結果 $t$ が異なることを要求する.二つ目の制約 $\bm{x}+\bm{r}^* \in A$ は,生成された敵対的サンプルが現実的な値域 $A$ に収まることを保証する.$A$ は,元データから各特徴量に対する最小値から最大値の集合であり,元データの分布を大きく崩さないようにするために用いられる.

% 目的関数について
LowProFoolでは上記の最適化問題を解くため,以下の目的関数 $g(\bm{r})$ を使用する.これを式(5)を定義する.

\autoequation{g(r) = L(\bm{x}+\bm{r}, t) + \lambda ||\bm{v} \odot \bm{r}||_2}

 目的関数は,二つの項から構成される.第一項 $L(\bm{x}+\bm{r}, t)$ は元データ $\bm{x}$ にノイズ $\bm{r}$ を付加したデータが誤分類させたい目標ラベル $t$ に分類させるようにする損失関数である.分類結果を変更させる役割がある.第二項 $\lambda ||\bm{v} \odot \bm{r}||_2$ は,特徴量重要度ベクトル $\bm{v}$ を用いてノイズの検知されにくさを制御している.ここで第二項にはバランスを調整するパラメータ $\lambda$ がある.
 これにより,元データの特徴量の重要度を考慮しつつ,検出されにくい敵対的サンプル $\bm{x}'$ を生成する.

次にLowProFoolのアルゴリズムを示す.アルゴリズムの各ステップは以下の通りである.
\begin{algorithm_step}
    \item[Step 1)] アルゴリズムで使用する変数の初期化を行う.ノイズベクトル $\bm{r}$ を
        ゼロベクトルで初期化する.また,初期サンプル $\bm{x}_0$ を元のサンプル $\bm{x}$ とする.
    
    \item[Step 2)] 最大反復回数 $N$ まで以下の計算を繰り返す.まず,前述の目的関数 
        $g(r) = L(\bm{x}+\bm{r}, t) + \lambda ||\bm{v} \odot \bm{r}||_2$ の勾配を計算する.
        次に,計算された勾配に基づいてノイズベクトル $\bm{r}$ を更新する.
        更新されたサンプルが有効な値域に収まるようクリッピングを行う.
    
    \item[Step 3)] 最適な敵対的サンプルの選択を行う.生成された敵対的サンプルの中から,
        元のサンプルとは異なるクラスに分類される($f(\bm{x}_i) \neq f(\bm{x}_0)$)ことと,
        検知されにくさの指標 $d_v(\bm{x}_i)$ が最小となるものを選ぶ.
    
    \item[Step 4)] 選択された敵対的サンプル $\bm{x}'$ を返す.
    \end{algorithm_step}


このアルゴリズムの特徴は,各反復において特徴量の重要度を考慮しながらノイズを更新する点にある.重要度の高い特徴量に対しては小さな摂動に抑えられ,重要度の低い特徴量により大きな摂動が許容される.これにより,分類結果を変更しつつも,検知されにくい敵対的サンプルの生成が可能となる.
また,クリッピング操作により,生成されるサンプルが常に有効な値域に収まることが保証される.例えば,年齢のような非負の特徴量が負の値を取ることを防ぐことができる.


\subsection{特徴量の重要度算出方法}

前述の通り,識別に寄与する特徴量を考慮した上で,ノイズを付加する.特徴量重要度 $\bm{v}$ の算出は,分類結果に対する各特徴量の相関係数を用いることが提案されていた.ピアソンの相関係数を用いて以下のように定義される.

\autoequation{\bm{v} = \cfrac{|\rho_{\bm{X},Y}|}{\|\rho_{\bm{X_i},Y}\|^2_2}}
ここで,$d$ は特徴量の次元数,$\|\rho_{\bm{X},Y}\|$ は $i$ 番目の特徴量 $\bm{X_i}$ と目的変数 $\bm{Y}$ の相関係数を示している.各特徴量と目的変数の関係性の強さを示す指標であり,値が大きいほどその特徴量が分類結果に寄与していることを意味する.また,分母は全ての特徴量の寄与度の二乗和の平方根を表し,全体の相関のスケールに対して分子を正規化している.これにより,特徴量重要度 $\bm{v}$ は,各特徴量が持つ相関の相対的な重要性を反映した値となる.この特性により,上記のアルゴリズムでは最小化を行うため,相関係数が大きい特徴量に対しては小さなノイズが付加されることになり,重要度を表現することができる.今回のデータセットによって生成された特徴量重要度を算出した図を下に示す.

\begin{figure}[H]
    \centering
    \includegraphics[width=0.8\textwidth]{images/従来手法_特徴量重要度.png}
    \caption{従来手法:各特徴量における重要度}
    \label{fig:default_method_feature_importance}
\end{figure}


\subsection{従来手法の課題}
従来手法を使用し,敵対的サンプルを生成した.生成された敵対的サンプルの一例を以下に示す.
\begin{table}[H]
    \centering
    \caption{元データと従来手法による敵対的サンプルの比較}
    \begin{tabular}{|c|c|c|} \hline
        特徴量 & 元データ & 従来手法  \\ \hline
        checking\_status & 0 & 0.09115 \\ \hline
        duration & 14 & 8.89098  \\ \hline
        credit\_amount & 8978 & 7153.84454 \\ \hline
        savings\_status & 0 & 0.083762\\ \hline
        employment & 4 & 3.947638 \\ \hline
        installment\_commitment & 1 & 1.071917 \\ \hline
        residence\_since & 4 & 4.000000  \\ \hline
        age & 45 & 44.413271 \\ \hline
        existing\_credits & 1 & 1.000000 \\ \hline
        num\_dependents & 1 & 1.000000 \\ \hline
        own\_telephone & 1 & 0.992310 \\ \hline
        foreign\_worker & 1 & 1.000000 \\ \hline
    \end{tabular}
\end{table}

生成された敵対的サンプルを確認すると以下の課題が見られた.

一つ目に特定の特徴量に対するノイズが集中してしまっていることをも挙げられる.特徴量重要度の算出方法において,相関係数が大きい特徴量に対してノイズを大きく避けるため,次に重要な特徴量へのノイズ集中してしまっている可能性がある.よって明らかに不自然に大きいノイズを抑えるような特徴量重要度の算出が重要になっている.
    

二つ目に,敵対的サンプルの出力が連続値であることが挙げられる.今回使用するデータセットは前述の通り特徴量が全て離散値である.敵対的サンプルが連続値を取ることで,離散値を持つ特徴量に対して生成された敵対的サンプルが現実的でない値をとってしまっている.このため,連続値の敵対的サンプルを離散値に変換する必要がある.


これらの課題に対して,次節では改良手法を提案する.

\section{提案手法}
%サンプルです.必要なければファイルごと削除してください.
この章では提案手法について説明します.
図や表をうまく使って,誰もが理解しやすい記述をしてください.

\subsection{提案手法の概要}

\subsection{重要度算出法の改善}

\subsection{出力データの離散化手法}

\section{実験}
% \subsection{実験概要}
% 提案手法によるより自然な敵対的サンプルが生成可能なのかどうか有効性を検証するため,敵対的サンプルの生成実験を行った.前述の通り,提案手法に示した重要度算出法と離散化手法を組み合わせ,敵対的サンプルを生成する.さらに,得られた敵対的サンプルの具体的な値について,適切か否かの確認を行う.
\subsection{実験条件}
今回の実験は,銀行のローンの審査システムについての機械学習モデルを想定し,誤分類を引き起こす敵対的サンプルの生成を行うことを目的とした.パラメータの設定は従来研究\cite{ballet2019imperceptible}に準拠している.式(5)で示したバランスを $\lambda=8.5$,最大繰り返し数が20000回,ニューラルネットワークの勾配降下法で使用する学習率を $\alpha=0.001$ とする.今回使用する機械学習モデルは,12個の特徴量を入力とし,2つのクラス(承認または拒否)を出力するニューラルネットワークである.このネットワークは,100個のノードを持つ隠れ層が6層にわたって配置された全結合型のモデルである.隠れ層はReLU関数で活性化され,出力層はバイナリクラス分類のためにシグモイド関数が使用されている.これらのモデルを図示したものを下に示す.

\begin{figure}[H]
    \centering
    \includegraphics[width=0.8\textwidth]{images/審査モデル.png}
    \caption{銀行のローン審査を行う機械学習モデルの構造}
    \label{fig:struct_model}
\end{figure}

また,学習においてBCELoss(Binary Cross Entropy Loss)が損失関数として使用している.モデルが各サンプルに対して予測した確率と実際のラベルとの誤差を計算し,二値分類における誤差を最小化する.最適化アルゴリズムにはAdamを使用し,学習率は $1.0 \times 10^{-4}$ に設定されている.バッチ学習を行い,バッチサイズは $N=100$ となっている.データを小分けにして学習を進め,予測精度はkeras.utils.to\_categoricalを用いてワンホットエンコーディングに変換し,各バッチで予測精度を計算した後,全データで平均を取っている.これによりモデルの分類精度を評価する.

\subsection{使用するデータセット}
これまでに説明した敵対的サンプルの生成手法を検証するため,実験に使用するデータセットについて説明する.

今回使用するデータは,OpenMLで公開されているUCI Machine Learnning RepositoryのGerman Credit Data Setである.\cite{credit-g}このデータセットは,ローンの信用リスクを予測する二値分類問題を扱っており,金融業界での実用的なモデル適用を想定した研究に適している.また,カテゴリカルデータと数値データの両方を含んでおり,敵対的サンプル生成手法の多様な適用可能性を評価するのに適した構造を持っている.このデータセットで使用した特徴量は,以下の通りである.

\begin{table}[H]
    \centering
    \caption{信用情報データセットの特徴量(ドイツマルク建)}
    \begin{tabular}{|l|l|l|}
        \hline
        特徴量名 & 属性 & 種類 \\ \hline
        checking\_status & 既存の当座預金口座のステータス & カテゴリ \\ \hline
        duration & ローン申請期間(月) & 数値 \\ \hline
        credit\_amount & クレジットの金額(申請する与信金額) & 数値 \\ \hline
        savings\_status & 貯蓄口座/債券のステータス & カテゴリ \\ \hline
        employment & 現在の雇用年数 & カテゴリ \\ \hline
        installment\_commitment & 可処分所得に対する分割払いの割合 & 数値 \\ \hline
        residence\_since & X年からの現在の居住地 & 数値 \\ \hline
        age & 年齢 & 数値 \\ \hline
        existing\_credits & この銀行の既存クレジット数 & 数値 \\ \hline
        num\_dependents & 扶養家族数 & 数値 \\ \hline
        own\_telephone & 電話 & カテゴリ \\ \hline
        foreign\_worker & 外国人労働者 & カテゴリ \\ \hline
        target & ローン承認 (true) or ローン却下(false) & カテゴリ \\ \hline
    \end{tabular}
    \label{tab:credit_g_features}
\end{table}

ここで上記の特徴量について追加の説明を行う.checking\_statusは顧客の当座預金口座を4つステータスを示しており,債務がある状態,200マルク以下の少額の預金がある状態,銀行の講座を保有していない状態,200マルク以上の多額の預金がある状態を指している.savings\_statusは貯蓄を4つのステータスで示しており,貯蓄が確認できない状態,貯蓄が100ドイツマルク以下の少額の貯蓄しかない状態,500マルク以下の少額から中程度の貯蓄がある状態,1000マルク以下の中程度の貯蓄がある状態,それ以上の高額な貯蓄がある状態を示している.また,表1で種類がカテゴリである特徴量についてはダミー変数化し離散値として扱う.また種類が数値である特徴量についても全て離散値であることがわかった.このデータセットは,ローンの信用リスクを予測するためのデータセットであるため,ローンの承認結果を予測する二値分類問題として扱う.

実験では,1000件のデータに対して正解データのバランスを保つようにデータをサンプリングする.取得した600件のデータのうち300件を学習データ,250件をテストデータ,残りを検証データに分割する.
まず,学習データでモデルの学習を行い,その後テストデータによるモデルの評価を行う.次にテストデータからランダムに10件取得し,それらをベースにノイズを加え敵対的サンプルを生成する.生成された敵対的サンプルに対して以下の評価指標を用いて評価を行う.

\subsection{評価指標}
提案手法の有効性の評価として以下の3つの指標を用い比較を行う.

一つ目に敵対的サンプルが元データと異なるクラスに分類されることを評価する成功率という指標である.ここで $\tilde{X}$ は $X$ をベースとして生成された敵対的サンプルの集合である.以下の式(9)に示す.
\autoequation{成功率 = \cfrac{|\tilde{X}|}{|X|}}
この指標が大きいほど敵対的サンプルが元データと異なるクラスに分類されることが多いことを示す.

二つ目に敵対的サンプル生成前のサンプルと生成後のサンプル間の距離を評価する指標である.まず,先ほどの敵対的サンプルとそのベースとなった元データのサンプルの集合について以下のように定義する.
以上をふまえ,以下の式(10)を平均距離として評価する.
\autoequation{平均距離 = \cfrac{1}{N} \sum^{N}_{i=1} \sqrt{\sum^{d}_{j=1}(x_{i, j}-\tilde{x}_{i, j})^2}}

これは,敵対的サンプルとその元になったデータの差分を求めている.これは敵対的サンプルを生成する過程で加えられたノイズの大きさを表しており,この指標が小さいほどノイズが小さく,元データに近いサンプルが生成されていることを示す.

三つ目に先ほどの平均距離に対して各特徴量の重要度で重み付けした指標である.これを重み距離と呼び,式(11)に示す.
\autoequation{重み距離 =  \cfrac{1}{N} \sum^{N}_{i=1} \sqrt{\sum^{d}_{j=1}( | x_{i, j}-\tilde{x}_{i, j}| \cdot v_j )^2}}
ここでは先ほどの平均距離で計算していた敵対的サンプルとその元になったデータの距離に対して,式(6)で定義した重要度を $\bm{v_j}$ とし,重み付けを行った距離を求めている.これは,重要度の高い特徴量にノイズを加えると,人間がその変更を知覚しやすくなることを距離で表現している.よって,敵対的サンプルの定義である人間が知覚しづらいノイズであるかどうかを判断するためこの指標を用いる.平均距離同様この値が小さいほどより自然で人間が知覚しずらいノイズを付加したことがわかる.

各指標を組み合わせて使用することで,生成した敵対的サンプルの「有効性」と「自然さ」の両面を総合的に評価することを目指す.

\subsection{実験結果}
実験結果について確認する.上記の評価指標に加えて,各手法における重要度特徴量と生成された敵対的サンプルの具体的な値についても確認する.

\subsubsection{成功率}
成功率について確認する.
\begin{table}[H]
    \centering
    \caption{実験結果:成功率}
    \begin{tabular}{|c|c|c|c|} \hline
        離散化 & 従来手法( $\bm{v}$ ) & 重要度算出( $\bm{v}_{\mathrm{raw}}$ ) & 重要度算出( $\bm{v}_{\mathrm{sqrt}}$ ) \\ \hline
        なし & 1.0 & 1.0 & 1.0 \\ \hline
        離散化(四捨五入) & 1.0 & 1.0 & 1.0 \\ \hline
        離散化(ランダム) & 1.0 & 1.0 & 1.0 \\ \hline
    \end{tabular}
\end{table}
すべての手法で10個の敵対的サンプルの生成ができた.これにより,提案手法による敵対的サンプルの精度の悪化などは発生していないことがわかる.よって,以降の平均距離,重み距離についてこの10個のサンプルを用いて評価を行っていく.

\subsubsection{平均距離}
平均距離について確認する.
\begin{table}[H]
    \centering
    \caption{実験結果:平均距離}
    \begin{tabular}{|c|c|c|c|} \hline
        離散化 & 従来手法( $\bm{v}$ ) & 重要度算出( $\bm{v}_{\mathrm{raw}}$ ) & 重要度算出( $\bm{v}_{\mathrm{sqrt}}$ ) \\ \hline
        なし & 0.377 & 0.260 & 0.173 \\ \hline
        離散化(四捨五入) & 0.399 & 0.264 & \textbf{0.163} \\ \hline
        離散化(ランダム) & 0.454 & 0.347 & 0.233 \\ \hline
    \end{tabular}
\end{table}
この結果から,従来手法よりも提案手法は,元データに近い敵対的サンプルを生成できていることがわかる.特に,重要度算出法( $\bm{v}_{\mathrm{sqrt}}$ )を用いた場合,最も小さい平均距離を示しており,元データに対する変更が最小限に抑えられていることが確認できる.離散化手法については四捨五入の方がよりノイズが小さくなっていることがわかる.これは,四捨五入による離散化が元データの特徴をより忠実に保持しているためと考えられる.さらに,重要度算出法( $\bm{v}_{\mathrm{sqrt}}$ )を用いた場合,四捨五入による離散化手法では,この実験で最小の平均距離を示している.これは,提案手法が元データの特徴を捉えつつ,適切なノイズを付加することで,より自然な敵対的サンプルを生成できていることを示している.また,従来手法( $\bm{v}$ )と比較して,提案手法( $\bm{v}_{\mathrm{raw}}$ および $\bm{v}_{\mathrm{sqrt}}$ )は,いずれの離散化手法においても平均距離が小さくなっている.これにより,提案手法が従来手法に比べて,より効果的に敵対的サンプルを生成できることが示されている.

\subsubsection{重み距離}
重み距離について確認する.
\begin{table}[H]
    \centering
    \caption{実験結果:重み距離}
    \begin{tabular}{|c|c|c|c|} \hline
        離散化 & 従来手法( $\bm{v}$ ) & 重要度算出( $\bm{v}_{\mathrm{raw}}$ ) & 重要度算出( $\bm{v}_{\mathrm{sqrt}}$ ) \\ \hline
        なし & 0.043 & 0.060 & 0.173\\ \hline
        離散化(四捨五入) & 0.044 & \textbf{0.048} & 0.061 \\ \hline
        離散化(ランダム) & 0.051 & 0.105 & 0.105 \\ \hline
    \end{tabular}
\end{table}
この結果から,提案手法は従来手法よりも良いスコアを出すことができなかった.特に,重要度算出( $\bm{v}_{\mathrm{sqrt}}$ )を用いた場合,重み距離が大きくなっていることが確認できる.これは,重要度算出法( $\bm{v}_{\mathrm{sqrt}}$ )が重要度の高い特徴量の値を小さくしてしまったため,ノイズの発生率を高めてしまったことが考えられる.一方で,重要度算出( $\bm{v}_{\mathrm{raw}}$ )における四捨五入による離散化手法では,従来手法に最も近い距離を示している.これは,四捨五入による離散化が元データの特徴を保持しつつ,適切なノイズを付加することで,重み距離を抑える効果があったと考えられる.また,離散化手法についても興味深い結果が得られた.離散化を行わない場合,提案手法( $\bm{v}_{\mathrm{raw}}$ )および( $\bm{v}_{\mathrm{sqrt}}$ )は従来手法よりも重み距離が大きくなっている.これは,連続値の敵対的サンプルが元データの分布から大きく逸脱しているためと考えられる.一方で,四捨五入による離散化手法を用いることで,重み距離が従来手法に近づいていることが確認できる.さらに,ランダム離散化手法では,提案手法( $\bm{v}_{\mathrm{raw}}$ )および( $\bm{v}_{\mathrm{sqrt}}$ )の重み距離が大きくなっている.これは,ランダム離散化が元データの特徴を保持する効果が低いためと考えられる.以上の結果から,提案手法は従来手法に比べて重み距離の面で劣る部分があるものの,重要度算出( $\bm{v}_{\mathrm{raw}}$ )と四捨五入による離散化手法の組み合わせが最も効果的であることが示された.この組み合わせにより,元データの特徴を保持しつつ,適切な敵対的サンプルを生成することができる.

\subsubsection{特徴量重要度}
従来手法($\bm{v}$),提案手法($\bm{v}_{\mathrm{raw}}$),提案手法($\bm{v}_{\mathrm{sqrt}}$)における特徴量重要度について以下の図に示す.
% 青色が従来手法($\bm{v}$),黄色が提案手法($\bm{v}_{\mathrm{raw}}$),緑色が提案手法($\bm{v}_{\mathrm{sqrt}}$)である.
\begin{figure}[H]
    \centering
    \includegraphics[width=0.8\textwidth]{images/実験_重要度算出の結果.png}
    \caption{各特徴量における重要度}
    \label{fig:importance}
\end{figure}
重要度について比較すると,従来手法では,checking\_statusに対する重要度が最も高く,次に重要度の高いdurationやcredit\_amountとの差が大きく離れていることがわかる.提案手法($\bm{v}_{\mathrm{raw}}$)は従来手法と比較し重要度の高い特徴量の値が特に小さくなっていることが確認できる.また,提案手法($\bm{v}_{\mathrm{sqrt}}$)では,提案手法($\bm{v}_{\mathrm{raw}}$)と比較して重要度の低い特徴量が大きく表現され特徴量間の差が小さく現れている.

結果をまとめると,提案手法($\bm{v}_{\mathrm{raw}}$)によって生成された敵対的サンプルは全て誤分類を引き起こす成功率100\%を示しており,平均距離では従来手法よりも小さいスコアを示し,重み距離では従来手法なみの知覚されにくさを表している.よって,自然な敵対的サンプルを生成できることができたということができる.

\subsubsection{生成された敵対的サンプル}
最後に,従来手法,提案手法によって生成された敵対的サンプルについて確認する.前述の使用するデータセットで説明したように,10件のサンプリングされたデータが元になっているため,それぞれ敵対的サンプルのベースとなるデータと比較を行う.今回は提案手法の実験の中で最も良い結果を示した重要度算出( $\bm{v}_{\mathrm{sqrt}}$ )における離散化(四捨五入)手法を用いて生成された敵対的サンプルについて比較を行う.

\begin{table}[H]
    \centering
    \caption{元データと敵対的サンプルの比較(id=268)}
    \begin{tabular}{|l|r|r|r|} \hline
        特徴量 & 元データ & 従来手法 & 提案手法 \\ \hline
        checking\_status & 0 & 0.09115 & 0 \\ \hline
        duration & 14 & 8.89098 & 15 \\ \hline
        credit\_amount & 8978 & 7153.84454 & 9169 \\ \hline
        savings\_status & 0 & 0.08376 & 0 \\ \hline
        employment & 4 & 3.94764 & 4 \\ \hline
        installment\_commitment & 1 & 1.07192 & 1 \\ \hline
        residence\_since & 4 & 4.00000 & 2 \\ \hline
        age & 45 & 44.41327 & 45 \\ \hline
        existing\_credits & 1 & 1.00000 & 1 \\ \hline
        num\_dependents & 1 & 1.00000 & 1 \\ \hline
        own\_telephone & 1 & 0.99231 & 1 \\ \hline
        foreign\_worker & 1 & 1.00000 & 1 \\ \hline
    \end{tabular}
\end{table}

全体として,提案手法では離散化により自然な敵対的サンプルになっていることがわかる.また,ノイズの大きさについては,idが268であるデータに注目するとchecking\_statusのノイズの回避が強くなり,durationやcredit\_amountに対してノイズが大きくなっている従来手法に対して,提案手法ではこれらの問題を抑えることができている.これにより,特定の特徴量に対する強いノイズを避ける傾向を抑え,なおかつ他の重要度の高い特徴量へのノイズを小さくすることができている.よって提案手法はより自然な敵対的サンプルを生成できることが示された.


\section{結論}
\subsection{まとめ}
本研究では,入力データの特徴とその重要度を考慮した敵対的サンプル生成手法について提案を行った.従来手法は,表形式データに対するノイズの付加を行う画期的な手法であったが,特定の特徴量に対して過剰なノイズが付与される問題が存在し,離散値を持つ表形式データに対して連続値の敵対的サンプルが生成される問題があった.そこで,特徴量の重要度算出方法を改善し,出力データの離散化を行うことで,より自然な敵対的サンプルの生成を実現した.実験結果から,提案手法により過度なノイズの付与を抑制でき,より自然な敵対的サンプルの生成が可能になった.
\subsection{今後の課題}
本研究では,使用したデータセットのみに適用を行った.実験結果しかし,他のデータセットなどに対して提案手法を適用することで,汎用性の高い手法であるかを検証する必要がある.また,提案手法のハイパーパラメータの調整や,他の敵対的サンプル生成手法との比較を行うことで,提案手法の有効性を評価する必要がある.


% 謝辞
\acknowledgements
% 謝辞の文面は過去の(豊田高専を含む)謝辞や,藤原の博士論文を参考にしてください.
% 博士論文の謝辞は長大ですが,卒業研究論文の謝辞をそんなに長くする必要は全くないと思います.
本論文の執筆にあたり,多くの方々から多大なるご支援とご指導を賜りましたことを,心より感謝申し上げます.
まず,指導教員の東京都市大学メディア情報学部情報システム学科 三川健太准教授には研究の最終段階に至るまで,度重なるご指導,ご鞭撻を賜りましたことを深く感謝いたします.
また,東京都市大学大学院環境情報学研究科環境情報学専攻 福田竜也氏には,研究の進行において多くの有益なアドバイスをいただきました.福田氏のご助言により研究をより前に進めることができました.心から感謝いたします.
さらに,日々のゼミ活動において多くの助言を賜りました,三川研究室の皆様にも感謝いたします.特に研究の環境構築において多大なるご協力をいただいた同研究室の佐竹航希さんに,心より感謝いたします.
皆様のご支援とご指導に,改めて感謝の意を表します.

% 参考文献
% BibTeX を使って参考文献のリストをつくって下さい.
% proceedings.bib: 雑誌名や国際会議の名称をマクロ化しています
% ref.bib: ここに参考文献を書くと良いと思います.元々博士論文に使った論文の情報が載っています.
% ref.bib以外のファイルに分けることも可能です.単純にカンマで区切ってファイル名(.bib除く)を
% \bibliographyの引数に追加してください.
\bibliography{proceedings,ref}

% 付録
% 付録には,本文に直接載せるべきではない,文字通り付録となる要素を載せます.
% 例えば,詳細な数式の展開であったり,アンケートや実験の詳細な(ほぼ完全な)結果だったりを載せます.
% 必要であれば以下の3行のコメントアウトを外して付録を加えてください.
\appendix
% サンプルです.全部削除してください.
\subsection*{ReLU関数}
ReLU関数は、入力が0より大きい場合はそのまま出力し、0以下の場合は0を出力する関数である。数式で表すと以下のようになる。

\subsection*{Softmax関数}



% 印刷量が膨大になる場合、コメントアウトを行う
%\section*{ソースコード}

% \input{src/Adverse.tex}
% \input{src/Metrics.tex}
% \begin{lstlisting}[style=jupyter, caption=敵対的サンプルを生成する流れの処理]
# In[1]:
# Misc
import random
import numpy as np
import pandas as pd
import tqdm
from tqdm import tqdm
from tqdm import tqdm_notebook
import math
import os
import time
import sys

# In[2]:
# Plotting
import matplotlib.pyplot as plt
import seaborn as sns

# In[3]:
# Sklearn
import sklearn
from sklearn.preprocessing import MinMaxScaler
from sklearn.datasets import fetch_openml
from sklearn.model_selection import train_test_split
from sklearn.metrics import accuracy_score

# In[4]:
# Pytorch
import torch
import torch.nn as nn
from torch.autograd import Variable
import torch.nn.functional as F
# Keras 
import keras

# In[5]:
# Helpers
from Adverse import lowProFool, deepfool
from Metrics import *

# In[6]:
# Retina display
%config InlineBackend.figure_format ='retina'
pd.set_option('display.max_columns', 500)
tqdm.pandas()
np.set_printoptions(suppress=True)

%load_ext autoreload
%autoreload 2

ccolors = ["#008ae9", "#ea004f"]
sns.set_palette(ccolors)
sns.palplot(sns.color_palette())

# In[7]:
SEED = 0
DATASET = 'credit-g'

# In[8]:
# Load initial dataset
df_orig, target, feature_names = get_df(DATASET)

# Balance dataset classes
df = balance_df(df_orig)

# Compute the bounds for clipping
bounds = get_bounds()

# Normalize the data
scaler, df, bounds = normalize(df, target, feature_names, bounds)

# Compute the weihts modelizing the expert's knowledge
weights = get_weights(df, target)
print("Weights", weights)

# Split df into train/test/valid
df_train, df_test, df_valid = split_train_test_valid()

# Build experimenation config
config = {'Dataset'     : 'credit-g',
         'MaxIters'     : 20000,
         'Alpha'        : 0.001,
         'Lambda'       : 8.5,
         'TrainData'    : df_train,
         'TestData'     : df_test,
         'ValidData'    : df_valid,
         'Scaler'       : scaler,
         'FeatureNames' : feature_names,
         'Target'       : target,
         'Weights'      : weights,
         'Bounds'       : bounds}

# Train neural network
model = get_model(config)
config['Model'] = model

# 推論の終了
model.eval()

# Compute accuracy on test set
y_true = df_test[target]
x_test = torch.FloatTensor(df_test[feature_names].values)
y_pred = model(x_test)

# y_predの確率をdataFrameに変換する処理
y_pred_numpy = y_pred.detach().numpy()
df_y_pred = pd.DataFrame(y_pred_numpy, columns=['0になる確率', '1になる確率'])    
# df_testのインデックスをリセットして、適切に整列させます
df_test_reset = df_test.reset_index()
# df_testとdf_y_predを列方向に連結します
df_combined_pred = pd.concat([df_test_reset, df_y_pred], axis=1)
print('df_combined_pred', df_combined_pred)

y_pred = np.argmax(y_pred.detach().numpy(), axis=1)
print("Accuracy score on test data", accuracy_score(y_true, y_pred))
    
# Sub sample
config['TestData'] = config['TestData'].sample(n=10, random_state = SEED)

# Generate adversarial examples
df_adv_lpf = gen_adv(config, 'LowProFool')
df_adv_df = gen_adv(config, 'Deepfool')
config['AdvData'] = {'LowProFool' : df_adv_lpf, 'Deepfool' : df_adv_df}

# Compute metrics
list_metrics = {'SuccessRate' : True,
                'iter_means': False,
                'iter_std': False,
                'normdelta_median': False,
                'normdelta_mean': True,
                'n_std': True,
                'weighted_median': False,
                'weighted_mean': True,
                'w_std': True,
                'mean_dists_at_org': False,
                'median_dists_at_org': False,
                'mean_dists_at_tgt': False,
                'mean_dists_at_org_weighted': True,
                'mdow_std': True,
                'median_dists_at_org_weighted': False,
                'mean_dists_at_tgt_weighted': True,
                'mdtw_std': True,
                'prop_same_class_arg_org': False,
                'prop_same_class_arg_adv': False}

all_metrics = get_metrics(config, list_metrics)
all_metrics = pd.DataFrame(all_metrics, columns=['Method'] + [k for k, v in list_metrics.items() if v])
all_metrics

# In[9]:
plot_ratios = []

m_lpf = all_metrics[all_metrics.Method == 'LowProFool']
m_df = all_metrics[all_metrics.Method =='Deepfool']

sr = m_lpf.SuccessRate.values / m_df.SuccessRate.values 
wm =  m_lpf.weighted_mean.values / m_df.weighted_mean.values 

plot_ratios.append([100*sr[0], 100*wm[0]])
plot_ratios = pd.DataFrame(plot_ratios, columns=['Success Rate Ratio', 'Mean Perturbation Ratio'])
plot_ratios['Dataset'] = 'German Credit'

f = plt.figure()
ax = plt.axes()
plot_ratios.plot(x='Dataset', kind='bar', legend=True, ax=ax)

for i, v in enumerate(plot_ratios['Success Rate Ratio'].values):
    ax.text(i - 0.2, v - 12 , str(v.round(1)) + '%', fontsize=16, color='white', weight='bold')
for i, v in enumerate(plot_ratios['Mean Perturbation Ratio'].values):
    ax.text(i + 0.062, v - 12, str(v.round(1)) + '%', fontsize=16, color='white', weight='bold')

plt.setp(ax.xaxis.get_majorticklabels(), rotation=0, ha='center')
ax.axhline(100, ls=':', c='grey')
ax.text(-0.49, 100 - 5, '100%')


ax.set_yticks([])
plt.tight_layout()
plt.show()

# In[10]:
# # 逆正規化
# df_adv_lpf = denormalize(scaler, df_adv_lpf, feature_names=feature_names)
# lowprofool
df_adv_lpf.to_csv('data/lowprofool_propose.csv')

# # 逆正規化
# df_adv_df = denormalize(scaler, df_adv_df, feature_names=feature_names)
# deepfool
df_adv_df.to_csv('data/deepfool_propose.csv')

# In[11]:
df_orig = pd.read_csv('data/df_orig.csv')
df_adv_lpf = pd.read_csv('data/lowprofool_propose.csv')
# 逆正規化
df_adv_lpf_denorm = denormalize(config['Scaler'], df_adv_lpf, feature_names=feature_names)

df_adv_df = pd.read_csv('data/deepfool_propose.csv')
# 逆正規化
df_adv_df_denorm = denormalize(config['Scaler'], df_adv_df, feature_names=feature_names)

# In[12]:
arrs = [685, 727, 30, 376, 66, 965, 963, 61, 282, 268]

for i in arrs:
    orig_row = df_orig[df_orig['Unnamed: 0'] == i].copy()
    adv_lpf_row = df_adv_lpf_denorm[df_adv_lpf_denorm['Unnamed: 0'] == i].copy()
    adv_df_row = df_adv_df_denorm[df_adv_df_denorm['Unnamed: 0'] == i].copy()
    
    # Rename columns to match the original format
    orig_row.columns = [col if col != 'Unnamed: 0' else 'id' for col in orig_row.columns]
    adv_lpf_row.columns = [col if col != 'Unnamed: 0' else 'id' for col in adv_lpf_row.columns]
    adv_df_row.columns = [col if col != 'Unnamed: 0' else 'id' for col in adv_df_row.columns]
    
    # Combine rows into a single DataFrame
    combined_df = pd.concat([orig_row, adv_lpf_row, adv_df_row], keys=['orig', 'adv_lpf', 'adv_df']).reset_index(level=0).rename(columns={'level_0': 'type'})
    
    # Save the DataFrame as df_output_{id}
    globals()[f'df_output_{i}'] = combined_df

# Example to access one of the generated DataFrames
df_output_685

# In[13]:
df_output_727

# In[14]:
df_output_30

# In[15]:
arrs = [685, 727, 30, 376, 66, 965, 963, 61, 282, 268]

for i in arrs:
    orig_row = df_orig[df_orig['Unnamed: 0'] == i].copy()
    adv_lpf_row = df_adv_lpf_denorm[df_adv_lpf_denorm['Unnamed: 0'] == i].copy()
    adv_df_row = df_adv_df_denorm[df_adv_df_denorm['Unnamed: 0'] == i].copy()
    
    # Rename columns to match the original format
    orig_row.columns = [col if col != 'Unnamed: 0' else 'id' for col in orig_row.columns]
    adv_lpf_row.columns = [col if col != 'Unnamed: 0' else 'id' for col in adv_lpf_row.columns]
    adv_df_row.columns = [col if col != 'Unnamed: 0' else 'id' for col in adv_df_row.columns]
    
    # Combine rows into a single DataFrame
    combined_df = pd.concat([orig_row, adv_lpf_row, adv_df_row], keys=['orig', 'adv_lpf', 'adv_df']).reset_index(level=0).rename(columns={'level_0': 'type'})
    
    # Round and convert to int
    combined_df.loc[:, combined_df.columns != 'type'] = combined_df.loc[:, combined_df.columns != 'type'].fillna(0).round().astype(int)
    
    # Save the DataFrame as df_output_{id}
    globals()[f'df_output_i_{i}'] = combined_df

# In[16]:
df_output_i_685

# In[17]:
df_output_i_727

# In[18]:
df_output_i_30

# In[19]:
df_output_i_376



\end{lstlisting} %すごい長いので途中で諦めた

\end{document}

