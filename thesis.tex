\documentclass[12pt,uplatex]{jsarticle}   % 日本語
\usepackage{tcu-thesis}
\usepackage{fancyhdr}

% オリジナルマクロ by Kyohei Fushida and Kenji Fujiwara
\newcommand{\li}{\item}
\newcommand{\ol}[1]{\begin{enumerate}#1\end{enumerate}}
\newcommand{\ul}[1]{\begin{itemize}#1\end{itemize}}
\newcommand{\dl}[1]{\begin{description}#1\end{description}}
\newcommand{\equ}[1]{\begin{equation}#1\end{equation}}
\newcommand{\eqenum}[1]{\begin{align}#1\end{align}}
\newcommand{\eqa}[1]{\begin{align}#1\end{align}}
\newcommand{\eqmul}[1]{\begin{equation}\begin{split}#1\end{split}\end{equation}}
\usepackage{xcolor}

%\newcommand{\figref}[1]{図\ref{#1}}
%\newcommand{\tabref}[1]{表\ref{#1}}
\renewcommand{\subsecref}[1]{\ref{#1}項}
\newcommand{\chapref}[1]{\ref{#1}章}

\definecolor{darkgreen}{rgb}{0, 0.5, 0} %下の色指定で使う
\definecolor{whitesmoke}{rgb}{0.99, 0.99, 0.99} %下の色指定で使う
\newcommand{\red}{\color{red}}%red
\newcommand{\blue}{\color{blue}}
\newcommand{\green}{\color{teal}}
\newcommand{\purple}{\color{violet}}%violet
\newcommand{\black}{\color{black}}
\newcommand{\ea}{\xspace\emph{et al.}\xspace}
\newcommand{\TODO}[1]{\nb{TODO:}{#1}}


\newcommand{\methods}{\#methods}
\newcommand{\colorrow}{\rowcolor[rgb]{0.851,0.851,0.851}}
\newcommand{\colorcel}{\cellcolor[rgb]{0.851,0.851,0.851}}
\def\et{\xspace et\ al.\xspace}
\def\ie{i.e.,\xspace}
\def\eg{e.g.,\xspace}
\newcommand{\nb}[2]{
    \fcolorbox{gray}{yellow}{\bfseries\sffamily\scriptsize#1}
    {\sf\small\textit{\textcolor{blue}{#2}}}
   }

\newtheorem{defi}{定義}
\newtheorem{hypo}{仮説}

% 必要なパッケージの読み込み
% とりあえず最低限使っておいて欲しいパッケージのみ書いています.
% 色々なパッケージがあります.藤原が過去に使ってたパッケージなどは博士論文のファイルを参照してください.
\usepackage[dvipdfmx]{graphicx}
\usepackage{cite} % 参考文献を [3,4]のようにまとめるのに便利.
\usepackage{url} % urlを記載するときに使う.

% 参考文献を引用された順で出力する
\bibliographystyle{junsrt}

% 学籍番号
\studentnumber{2172010}

% 日本語題目
\title{入力データの構造を考慮したLowProFoolアルゴリズムによる\\敵対的サンプルの生成に関する研究}

% 英語題目
\etitle{Generation of Adversarial Samples Using Modified LowProFool Algorithm with Consideration of Input Data Structure}

% 日本語氏名(姓と名の間に空白(半角)を入れて下さい)
\author{有馬 祥太}

% 欧文氏名(first name, last name の順に記入し、先頭文字のみを大文字にする。)
\eauthor{Shota Arima}

% 専攻
\department{\media}
\course{\jsys}

% 論文提出年月日
\syear{2025}
\smonth{1}
\sday{30}

% キーワード5〜6個 (in LaTeX)
\keywords{敵対的サンプル,表形式データ,LowProFoolアルゴリズム,データ離散化,特徴量重要度}

% 5 or 6 Keywords (in LaTeX)
\ekeywords{Adversarial Samples, Table Data, LowProFool Algorithm, Data discretization, Importance of Features}

% 内容梗概 (in LaTeX)
%   Abstractは必須です。
%   サンプルほど長々と書く必要はありません.
%   注: 行の先頭が\\で始まらないようにすること。
\abstract{%
この研究では,機械学習モデルに対する敵対的サンプル生成において,表形式データのおける敵対的サンプルの特性を考慮した改良手法を提案しています。
従来のLowProFoolアルゴリズムでは特定の特徴量に対して過度にノイズが付与されてしまう可能性と離散値を持つ表形式データに対して連続値の敵対的サンプルを生成してしまう課題がありました。
そこで本研究では,特徴量の重要度算出方法を改善し,出力データの離散化を行うことで,より自然な敵対的サンプルの生成を実現しました。
実験結果から,提案手法により過度なノイズの付与を抑制でき,より現実的な敵対的サンプルの生成が可能になりました。
}

%%%%%%%%%%%%%%%%%%%%%%%%% document starts here %%%%%%%%%%%%%%%%%%%%%%%%%%%%

\begin{document}
%
% 表紙 および アブストラクト
%
\titlepage
\jabstractpage
%
% 目次
%
\tableofcontents
\newpage

% 以降本文
\pagenumbering{arabic} % ページ番号をアラビア数字に戻す

% この章構成はあくまで一例です.
% \section:章
% \subsection: 節
% \subsubsection: 項
% として使ってください.
% 研究の内容によって自分が適切だと思う章構成を考えて下さい.
% もちろん構成について藤原と相談するのも良いと思います.
\section{はじめに}
%サンプルです.必要なければファイルごと削除してください.
「はじめに」では研究の背景を述べつつ研究の目的を述べるのが一般的です.
仰々しい論文の場合は論文の構成(以降の章構成)を説明したりもします.

最初は世の中(といっても情報系分野ぐらいで十分ですが)全体の背景を述べつつ,
徐々に的を絞ってどうして自身の研究を行う必要があるのかを上手く説明できると素晴らしいです.
研究の背景となる課題や既存研究を紹介する際には必ず参考文献を引用するようにしましょう.

「はじめに」は他の章に比べて非常に作成が難しいです.
どういう流れで書けばいいのか困った際は指導教員に聞くのもよいと思いまし,とりあえず書ける章(自分が実際に手を動かした章)から書いていくのも良いと思います.

\subsection{第一章のタイトルについて}
「はじめに」,「序論」,「緒言」など色々なタイトルを付けることができます.
個人の好みで章のタイトルを決めて貰ってよいですが,例えば「はじめに」であれば「おわりに」,
「序論」は「結論」,「緒言は「結言」など,それぞれ対応する言葉があるので気をつけましょう.

\subsection{論文を書く際の注意事項}
\begin{itemize}
    \item 句読点は全角のカンマとピリオドである「,」と「.」を使って下さい.卒論執筆期間中は一時的にIMEの設定を変更して常にカンマとピリオドが使用されるようにしておくと安全だと思います.
    \item 表記揺れに注意しましょう.
    \item ですます調はNGです.
    \item 未定義語は使わない.一般的でない用語が出てくるタイミングで必ず説明を入れる.
\end{itemize}

\section{関連研究・用語}
%サンプルです.必要なければファイルごと削除してください.
この章では自分の研究に関連する既存研究,自身の研究の説明に必要となりそうな技術について紹介してください.
例えば,micro:bitを扱う研究などではmicro:bitとはそもそも何なのかを説明する節を追加しても良いです.
卒論等では,そもそも自身の研究に関する背景知識がそれほど無い人でも何をしているのかが分かるように,背景となる研究・技術をしっかり自身の言葉でまとめてください.
ここでWeb等に誰かが書いた内容を丸々コピーしていた場合は,最悪卒論として受理しないという可能性もあるので気をつけてください.
ある程度subsectionを作成して,構造化できると良いと思います.

\section{提案手法}
%サンプルです.必要なければファイルごと削除してください.
この章では提案手法について説明します.
図や表をうまく使って,誰もが理解しやすい記述をしてください.

\section{実験}
%サンプルです.必要なければファイルごと削除してください.
この章では,提案手法を実際に適用する実験について説明してください.
手法の細かいパラメータ設定などは提案手法などではなく実験の章で説明することが多いです.
あくまで,提案手法はより広い範囲に手法を適用できるような抽象的な手順を書くのに留めましょう.

\section{実験結果}
\input{contents/result}

\section{考察}
%サンプルです.必要なければファイルごと削除してください.
結果の章で説明した実験結果について細かく考察を行ってください.
考察は感想ではありません.どういうことを書けば良いかは \url{https://www.enago.jp/academy/results-and-discussion/} など,色々なサイトや論文の書き方を説明している本を参考にしてみてください.

\section{妥当性の脅威}
%サンプルです.必要なければファイルごと削除してください.

\section{おわりに}
\input{contents/conclusion}


% 謝辞
\acknowledgements
% 謝辞の文面は過去の(豊田高専を含む)謝辞や,藤原の博士論文を参考にしてください.
% 博士論文の謝辞は長大ですが,卒業研究論文の謝辞をそんなに長くする必要は全くないと思います.
色々と感謝します.


% 参考文献
% BibTeX を使って参考文献のリストをつくって下さい.
% proceedings.bib: 雑誌名や国際会議の名称をマクロ化しています
% ref.bib: ここに参考文献を書くと良いと思います.元々博士論文に使った論文の情報が載っています.
% ref.bib以外のファイルに分けることも可能です.単純にカンマで区切ってファイル名(.bib除く)を
% \bibliographyの引数に追加してください.
\bibliography{proceedings,ref}

% 付録
% 付録には,本文に直接載せるべきではない,文字通り付録となる要素を載せます.
% 例えば,詳細な数式の展開であったり,アンケートや実験の詳細な(ほぼ完全な)結果だったりを載せます.
% 必要であれば以下の3行のコメントアウトを外して付録を加えてください.
%\appendix
%% サンプルです.全部削除してください.
\subsection*{ReLU関数}
ReLU関数は、入力が0より大きい場合はそのまま出力し、0以下の場合は0を出力する関数である。数式で表すと以下のようになる。

\subsection*{Softmax関数}



% 印刷量が膨大になる場合、コメントアウトを行う
%\section*{ソースコード}

% \input{src/Adverse.tex}
% \input{src/Metrics.tex}
% \begin{lstlisting}[style=jupyter, caption=敵対的サンプルを生成する流れの処理]
# In[1]:
# Misc
import random
import numpy as np
import pandas as pd
import tqdm
from tqdm import tqdm
from tqdm import tqdm_notebook
import math
import os
import time
import sys

# In[2]:
# Plotting
import matplotlib.pyplot as plt
import seaborn as sns

# In[3]:
# Sklearn
import sklearn
from sklearn.preprocessing import MinMaxScaler
from sklearn.datasets import fetch_openml
from sklearn.model_selection import train_test_split
from sklearn.metrics import accuracy_score

# In[4]:
# Pytorch
import torch
import torch.nn as nn
from torch.autograd import Variable
import torch.nn.functional as F
# Keras 
import keras

# In[5]:
# Helpers
from Adverse import lowProFool, deepfool
from Metrics import *

# In[6]:
# Retina display
%config InlineBackend.figure_format ='retina'
pd.set_option('display.max_columns', 500)
tqdm.pandas()
np.set_printoptions(suppress=True)

%load_ext autoreload
%autoreload 2

ccolors = ["#008ae9", "#ea004f"]
sns.set_palette(ccolors)
sns.palplot(sns.color_palette())

# In[7]:
SEED = 0
DATASET = 'credit-g'

# In[8]:
# Load initial dataset
df_orig, target, feature_names = get_df(DATASET)

# Balance dataset classes
df = balance_df(df_orig)

# Compute the bounds for clipping
bounds = get_bounds()

# Normalize the data
scaler, df, bounds = normalize(df, target, feature_names, bounds)

# Compute the weihts modelizing the expert's knowledge
weights = get_weights(df, target)
print("Weights", weights)

# Split df into train/test/valid
df_train, df_test, df_valid = split_train_test_valid()

# Build experimenation config
config = {'Dataset'     : 'credit-g',
         'MaxIters'     : 20000,
         'Alpha'        : 0.001,
         'Lambda'       : 8.5,
         'TrainData'    : df_train,
         'TestData'     : df_test,
         'ValidData'    : df_valid,
         'Scaler'       : scaler,
         'FeatureNames' : feature_names,
         'Target'       : target,
         'Weights'      : weights,
         'Bounds'       : bounds}

# Train neural network
model = get_model(config)
config['Model'] = model

# 推論の終了
model.eval()

# Compute accuracy on test set
y_true = df_test[target]
x_test = torch.FloatTensor(df_test[feature_names].values)
y_pred = model(x_test)

# y_predの確率をdataFrameに変換する処理
y_pred_numpy = y_pred.detach().numpy()
df_y_pred = pd.DataFrame(y_pred_numpy, columns=['0になる確率', '1になる確率'])    
# df_testのインデックスをリセットして、適切に整列させます
df_test_reset = df_test.reset_index()
# df_testとdf_y_predを列方向に連結します
df_combined_pred = pd.concat([df_test_reset, df_y_pred], axis=1)
print('df_combined_pred', df_combined_pred)

y_pred = np.argmax(y_pred.detach().numpy(), axis=1)
print("Accuracy score on test data", accuracy_score(y_true, y_pred))
    
# Sub sample
config['TestData'] = config['TestData'].sample(n=10, random_state = SEED)

# Generate adversarial examples
df_adv_lpf = gen_adv(config, 'LowProFool')
df_adv_df = gen_adv(config, 'Deepfool')
config['AdvData'] = {'LowProFool' : df_adv_lpf, 'Deepfool' : df_adv_df}

# Compute metrics
list_metrics = {'SuccessRate' : True,
                'iter_means': False,
                'iter_std': False,
                'normdelta_median': False,
                'normdelta_mean': True,
                'n_std': True,
                'weighted_median': False,
                'weighted_mean': True,
                'w_std': True,
                'mean_dists_at_org': False,
                'median_dists_at_org': False,
                'mean_dists_at_tgt': False,
                'mean_dists_at_org_weighted': True,
                'mdow_std': True,
                'median_dists_at_org_weighted': False,
                'mean_dists_at_tgt_weighted': True,
                'mdtw_std': True,
                'prop_same_class_arg_org': False,
                'prop_same_class_arg_adv': False}

all_metrics = get_metrics(config, list_metrics)
all_metrics = pd.DataFrame(all_metrics, columns=['Method'] + [k for k, v in list_metrics.items() if v])
all_metrics

# In[9]:
plot_ratios = []

m_lpf = all_metrics[all_metrics.Method == 'LowProFool']
m_df = all_metrics[all_metrics.Method =='Deepfool']

sr = m_lpf.SuccessRate.values / m_df.SuccessRate.values 
wm =  m_lpf.weighted_mean.values / m_df.weighted_mean.values 

plot_ratios.append([100*sr[0], 100*wm[0]])
plot_ratios = pd.DataFrame(plot_ratios, columns=['Success Rate Ratio', 'Mean Perturbation Ratio'])
plot_ratios['Dataset'] = 'German Credit'

f = plt.figure()
ax = plt.axes()
plot_ratios.plot(x='Dataset', kind='bar', legend=True, ax=ax)

for i, v in enumerate(plot_ratios['Success Rate Ratio'].values):
    ax.text(i - 0.2, v - 12 , str(v.round(1)) + '%', fontsize=16, color='white', weight='bold')
for i, v in enumerate(plot_ratios['Mean Perturbation Ratio'].values):
    ax.text(i + 0.062, v - 12, str(v.round(1)) + '%', fontsize=16, color='white', weight='bold')

plt.setp(ax.xaxis.get_majorticklabels(), rotation=0, ha='center')
ax.axhline(100, ls=':', c='grey')
ax.text(-0.49, 100 - 5, '100%')


ax.set_yticks([])
plt.tight_layout()
plt.show()

# In[10]:
# # 逆正規化
# df_adv_lpf = denormalize(scaler, df_adv_lpf, feature_names=feature_names)
# lowprofool
df_adv_lpf.to_csv('data/lowprofool_propose.csv')

# # 逆正規化
# df_adv_df = denormalize(scaler, df_adv_df, feature_names=feature_names)
# deepfool
df_adv_df.to_csv('data/deepfool_propose.csv')

# In[11]:
df_orig = pd.read_csv('data/df_orig.csv')
df_adv_lpf = pd.read_csv('data/lowprofool_propose.csv')
# 逆正規化
df_adv_lpf_denorm = denormalize(config['Scaler'], df_adv_lpf, feature_names=feature_names)

df_adv_df = pd.read_csv('data/deepfool_propose.csv')
# 逆正規化
df_adv_df_denorm = denormalize(config['Scaler'], df_adv_df, feature_names=feature_names)

# In[12]:
arrs = [685, 727, 30, 376, 66, 965, 963, 61, 282, 268]

for i in arrs:
    orig_row = df_orig[df_orig['Unnamed: 0'] == i].copy()
    adv_lpf_row = df_adv_lpf_denorm[df_adv_lpf_denorm['Unnamed: 0'] == i].copy()
    adv_df_row = df_adv_df_denorm[df_adv_df_denorm['Unnamed: 0'] == i].copy()
    
    # Rename columns to match the original format
    orig_row.columns = [col if col != 'Unnamed: 0' else 'id' for col in orig_row.columns]
    adv_lpf_row.columns = [col if col != 'Unnamed: 0' else 'id' for col in adv_lpf_row.columns]
    adv_df_row.columns = [col if col != 'Unnamed: 0' else 'id' for col in adv_df_row.columns]
    
    # Combine rows into a single DataFrame
    combined_df = pd.concat([orig_row, adv_lpf_row, adv_df_row], keys=['orig', 'adv_lpf', 'adv_df']).reset_index(level=0).rename(columns={'level_0': 'type'})
    
    # Save the DataFrame as df_output_{id}
    globals()[f'df_output_{i}'] = combined_df

# Example to access one of the generated DataFrames
df_output_685

# In[13]:
df_output_727

# In[14]:
df_output_30

# In[15]:
arrs = [685, 727, 30, 376, 66, 965, 963, 61, 282, 268]

for i in arrs:
    orig_row = df_orig[df_orig['Unnamed: 0'] == i].copy()
    adv_lpf_row = df_adv_lpf_denorm[df_adv_lpf_denorm['Unnamed: 0'] == i].copy()
    adv_df_row = df_adv_df_denorm[df_adv_df_denorm['Unnamed: 0'] == i].copy()
    
    # Rename columns to match the original format
    orig_row.columns = [col if col != 'Unnamed: 0' else 'id' for col in orig_row.columns]
    adv_lpf_row.columns = [col if col != 'Unnamed: 0' else 'id' for col in adv_lpf_row.columns]
    adv_df_row.columns = [col if col != 'Unnamed: 0' else 'id' for col in adv_df_row.columns]
    
    # Combine rows into a single DataFrame
    combined_df = pd.concat([orig_row, adv_lpf_row, adv_df_row], keys=['orig', 'adv_lpf', 'adv_df']).reset_index(level=0).rename(columns={'level_0': 'type'})
    
    # Round and convert to int
    combined_df.loc[:, combined_df.columns != 'type'] = combined_df.loc[:, combined_df.columns != 'type'].fillna(0).round().astype(int)
    
    # Save the DataFrame as df_output_{id}
    globals()[f'df_output_i_{i}'] = combined_df

# In[16]:
df_output_i_685

# In[17]:
df_output_i_727

# In[18]:
df_output_i_30

# In[19]:
df_output_i_376



\end{lstlisting} %すごい長いので途中で諦めた

\end{document}

